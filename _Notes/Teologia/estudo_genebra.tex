%%%%%%%%%%%%%%%%%%%%%%%%%%%%%%%%%%%%%%%%%%%%%%%%%%%
%% LaTeX book template                           %%
%% Author:  Amber Jain (http://amberj.devio.us/) %%
%% License: ISC license                          %%
%%%%%%%%%%%%%%%%%%%%%%%%%%%%%%%%%%%%%%%%%%%%%%%%%%%

\documentclass[a4paper,11pt,oneside]{book}

% pacotes utilizados
\usepackage[T1]{fontenc}
\usepackage[utf8]{inputenc}
\usepackage{lmodern}
\usepackage{hyperref}
\usepackage{graphicx}
\usepackage[portuguese]{babel}
\usepackage{amsfonts}
\usepackage[usenames,dvipsnames]{xcolor}
\usepackage{mathtools}
\usepackage{amssymb} % alguns simbolos matematicos
\usepackage{mathrsfs} % letras cursivas
\usepackage{listings} % coding examples in latex file
\usepackage{xcolor} % changing colors
\usepackage{amsthm} % definir estilo do teorema
\usepackage[hang,flushmargin]{footmisc} % remove footnote's identation
\usepackage{cancel} % para poder colocar o tracinho de cancelamento
\usepackage{tikz} % to draw Venn's diagrams
\usepackage{amsmath} % to write a function by cases
\usepackage{graphicx} % to set a images directory

\graphicspath{{./images/}}

% dark theme no pdf
\pagecolor[rgb]{0.1,0.1,0.1} %black
\color[rgb]{0.9,0.9,0.9} %grey

% criando o modelo de definicoes/teoremas/fatos/demonstracoes
\theoremstyle{definition}

\newtheoremstyle{break}% name
  {10pt}%         Space above, empty = `usual value'
  {10pt}%         Space below
  {}% Body font
  {}%         Indent amount (empty = no indent, \parindent = para indent)
  {\bfseries}% Thm head font
  {}%        Punctuation after thm head
  {\newline}% Space after thm head: \newline = linebreak
  {}%         Thm head spec

\theoremstyle{break}

% definindo as categorias de formalidade
\newtheorem{definition}{Definição}[section]
\newtheorem{fact}{Fato}[section]
\newtheorem{demonstration}{Demonstração}[section]
\newtheorem{theorem}{Teorema}

% tirando identação dos paragrafos
\setlength{\parindent}{0ex}

% setting das cores quando usar codigo python
\definecolor{codegreen}{rgb}{0,0.6,0}
\definecolor{codegray}{rgb}{0.5,0.5,0.5}
\definecolor{codepurple}{rgb}{0.58,0,0.82}
\definecolor{backcolour}{rgb}{0.95,0.95,0.92}

\lstdefinestyle{mystyle}{
    backgroundcolor=\color{backcolour},   
    commentstyle=\color{codegreen},
    keywordstyle=\color{magenta},
    numberstyle=\tiny\color{codegray},
    stringstyle=\color{codepurple},
    basicstyle=\ttfamily\footnotesize,
    breakatwhitespace=false,         
    breaklines=true,                 
    captionpos=b,                    
    keepspaces=true,                 
    numbers=left,                    
    numbersep=5pt,                  
    showspaces=false,                
    showstringspaces=false,
    showtabs=false,                  
    tabsize=2
}

\lstset{style=mystyle}

% dedicatoria 
% Source: http://www.tug.org/pipermail/texhax/2010-June/
\newenvironment{dedication}
{
   \cleardoublepage
   \thispagestyle{empty}
   \vspace*{\stretch{1}}
   \hfill\begin{minipage}[t]{0.66\textwidth}
   \raggedright
}
{
   \end{minipage}
   \vspace*{\stretch{3}}
   \clearpage
}

% Chapter quote at the start of chapter        %
% Source: http://tex.stackexchange.com/a/53380 %
\makeatletter
\renewcommand{\@chapapp}{}% Not necessary...
\newenvironment{chapquote}[2][2em]
  {\setlength{\@tempdima}{#1}%
   \def\chapquote@author{#2}%
   \parshape 1 \@tempdima \dimexpr\textwidth-2\@tempdima\relax%
   \itshape}
  {\par\normalfont\hfill--\ \chapquote@author\hspace*{\@tempdima}\par\bigskip}
\makeatother


%%%%%%%%%%%%%%%%%%%%%%%%%%%%%%%%%%%%%%%%%%%%%%%%%%%
% First page of book which contains 'stuff' like: %
%  - Book title, subtitle                         %
%  - Book author name                             %
%%%%%%%%%%%%%%%%%%%%%%%%%%%%%%%%%%%%%%%%%%%%%%%%%%%
% Book's title and subtitle
\title{\Huge \textbf{Bíblia de Estudo de Genebra} \\ 
\Large Segunda Edição \\
\huge Sociedade Bíblica do Brasil}

% Author
\author{
\textsc{Resumo e Adaptação por:} \\
\textsc{Bruno de M. Ruas}
}

\begin{document}

\frontmatter
\maketitle

\tableofcontents
%\listoffigures
%\listoftables

\mainmatter


%%%%%%%%%%%%%%%%%%%%%%%% PART %%%%%%%%%%%%%%%%%%%%%%%%
\part{Antigo Testamento}

%%%%%%%%%%%%%%%%%%%%%%%% CHAPTER %%%%%%%%%%%%%%%%%%%%%%%%
\chapter{Gênesis}

\begin{chapquote}{página 3}
	``The theory of sets is a language that is perfectly suited to describing and explaning all types of mathematical structures.''
\end{chapquote}

%%%%%%%%%%%%%%%%%%%%%%%% CHAPTER %%%%%%%%%%%%%%%%%%%%%%%%
\chapter{Êxodo}
\chapter{Levítico}
\chapter{Números}
\chapter{Deuteronômio}
\chapter{Josué}
\chapter{Juízes}
\chapter{Rute}
\chapter{1 Samuel}
\chapter{2 Samuel}
\chapter{1 Reis}
\chapter{2 Reis}
\chapter{1 Crônicas}
\chapter{2 Crônicas}
\chapter{Esdras}
\chapter{Neemias}
\chapter{Ester}
\chapter{Jó}
\chapter{Salmos}
\chapter{Provérbios}
\chapter{Eclesiastes}
\chapter{Cantares}
\chapter{Isaías}
\chapter{Jeremias}
\chapter{Lamentações de Jeremias}
\chapter{Ezequiel}
\chapter{Daniel}
\chapter{Oséias}
\chapter{Joel}
\chapter{Amós}
\chapter{Obadias}
\chapter{Jonas}
\chapter{Miquéias}
\chapter{Naum}
\chapter{Habacuque}
\chapter{Sofonias}
\chapter{Ageu}
\chapter{Zacarias}
\chapter{Malaquias}

%%%%%%%%%%%%%%%%%%%%%%%% PART %%%%%%%%%%%%%%%%%%%%%%%%
\part{Novo Testamento}

%%%%%%%%%%%%%%%%%%%%%%%% CHAPTER %%%%%%%%%%%%%%%%%%%%%%%%
\chapter{Mateus}
\chapter{Marcos}
\chapter{Lucas}
\chapter{João}
\chapter{Atos}
\chapter{Romanos}

\begin{chapquote}{página 3}
	\textbf{Autor: } Apóstolo Paulo \\
	\textbf{Data e Ocasião: } Pouco antes de visitar Jerusalém. Provavelmente enquanto estava em Corinto.
\end{chapquote}


\chapter{1 Coríntios}
\chapter{2 Coríntios}
\chapter{Gálatas}
\chapter{Efésios}
\chapter{Filipenses}
\chapter{Colossenses}
\chapter{1 Tessalonicenses}
\chapter{2 Tessalonicenses}
\chapter{1 Timóteo}
\chapter{2 Timóteo}
\chapter{Tito}
\chapter{Filemon}
\chapter{Hebreus}
\chapter{Tiago}
\chapter{1 Pedro}
\chapter{2 Pedro}
\chapter{1 João}
\chapter{2 João}
\chapter{3 João}
\chapter{Judas}
\chapter{Apocalipse}

\end{document}
