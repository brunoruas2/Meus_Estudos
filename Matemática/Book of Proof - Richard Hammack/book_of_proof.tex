%%%%%%%%%%%%%%%%%%%%%%%%%%%%%%%%%%%%%%%%%%%%%%%%%%%
%% LaTeX book template                           %%
%% Author:  Amber Jain (http://amberj.devio.us/) %%
%% License: ISC license                          %%
%%%%%%%%%%%%%%%%%%%%%%%%%%%%%%%%%%%%%%%%%%%%%%%%%%%

\documentclass[a4paper,11pt]{book}
\usepackage[T1]{fontenc}
\usepackage[utf8]{inputenc}
\usepackage{lmodern}
\usepackage{hyperref}
\usepackage{graphicx}
\usepackage[portuguese]{babel}

% tirando identação dos paragrafos
\setlength{\parindent}{0ex}

% coding examples in latex file
\usepackage{listings}
\usepackage{xcolor}

\definecolor{codegreen}{rgb}{0,0.6,0}
\definecolor{codegray}{rgb}{0.5,0.5,0.5}
\definecolor{codepurple}{rgb}{0.58,0,0.82}
\definecolor{backcolour}{rgb}{0.95,0.95,0.92}

\lstdefinestyle{mystyle}{
    backgroundcolor=\color{backcolour},   
    commentstyle=\color{codegreen},
    keywordstyle=\color{magenta},
    numberstyle=\tiny\color{codegray},
    stringstyle=\color{codepurple},
    basicstyle=\ttfamily\footnotesize,
    breakatwhitespace=false,         
    breaklines=true,                 
    captionpos=b,                    
    keepspaces=true,                 
    numbers=left,                    
    numbersep=5pt,                  
    showspaces=false,                
    showstringspaces=false,
    showtabs=false,                  
    tabsize=2
}

\lstset{style=mystyle}

%%%%%%%%%%%%%%%%%%%%%%%%%%%%%%%%%%%%%%%%%%%%%%%%%%%%%%%%%%%%%%%%%%%%%%%%%%%%%%%%
% 'dedication' environment: To add a dedication paragraph at the start of book %
% Source: http://www.tug.org/pipermail/texhax/2010-June/015184.html            %
%%%%%%%%%%%%%%%%%%%%%%%%%%%%%%%%%%%%%%%%%%%%%%%%%%%%%%%%%%%%%%%%%%%%%%%%%%%%%%%%
\newenvironment{dedication}
{
   \cleardoublepage
   \thispagestyle{empty}
   \vspace*{\stretch{1}}
   \hfill\begin{minipage}[t]{0.66\textwidth}
   \raggedright
}
{
   \end{minipage}
   \vspace*{\stretch{3}}
   \clearpage
}

%%%%%%%%%%%%%%%%%%%%%%%%%%%%%%%%%%%%%%%%%%%%%%%%
% Chapter quote at the start of chapter        %
% Source: http://tex.stackexchange.com/a/53380 %
%%%%%%%%%%%%%%%%%%%%%%%%%%%%%%%%%%%%%%%%%%%%%%%%
\makeatletter
\renewcommand{\@chapapp}{}% Not necessary...
\newenvironment{chapquote}[2][2em]
  {\setlength{\@tempdima}{#1}%
   \def\chapquote@author{#2}%
   \parshape 1 \@tempdima \dimexpr\textwidth-2\@tempdima\relax%
   \itshape}
  {\par\normalfont\hfill--\ \chapquote@author\hspace*{\@tempdima}\par\bigskip}
\makeatother


%%%%%%%%%%%%%%%%%%%%%%%%%%%%%%%%%%%%%%%%%%%%%%%%%%%
% First page of book which contains 'stuff' like: %
%  - Book title, subtitle                         %
%  - Book author name                             %
%%%%%%%%%%%%%%%%%%%%%%%%%%%%%%%%%%%%%%%%%%%%%%%%%%%
% Book's title and subtitle
\title{\Huge \textbf{Book of Proof} \\ 
\huge Third Edition}
% Author
\author{\textsc{Richard Hammack}}


\begin{document}

\frontmatter
\maketitle

\tableofcontents
%\listoffigures
%\listoftables

\mainmatter

%%%%%%%%%%%%%%%%%%%%%%%%%%%%%%%%%%%%%%%%%%%%%%%%%%%
%                 CHAPTER                         %
%%%%%%%%%%%%%%%%%%%%%%%%%%%%%%%%%%%%%%%%%%%%%%%%%%%
\chapter{Conjuntos}


\begin{chapquote}{página 3}
	``The theory of sets is a language that is perfectly suited to describing and explaning all types of mathematical structures.''
\end{chapquote}


\section{Introdução}
Um \textbf{conjunto} (set) é uma lista de \textbf{elementos}. Normalmente denotados por uma letra maiúscula. Por exemplo:

\begin{center}
	$ A = \{1 , 2 , 3 , 4 , ... \} $
\end{center}

Dois sets $A$ e $B$ são \textbf{iguais} se possuírem exatamente os mesmos elementos. Não importando a ordem desses elementos dentro de cada set.
\\
\\
Vamos definir um token para sinalizar se um determinado elemento $x$ pertence ou não a um determinado set qualquer $A$. Para tal relação usaremos o símbolo "$\in$" se $x$ for um elemento de $A$ ou, caso contrário, usaremos "$\notin$" se $x$ não for um elemento de $A$.
\\
\\
É provável que, em algum momento, seja necessário contar a quantidade de elementos em um dado set qualquer $A$. Chamaremos essa relação de \textbf{cardinalidade} ou \textbf{tamanho} do set $A$. O token usado será duas barras em volta do set do seguinte modo: "$|A|$".
\\
\\
A partir dessas duas relações já podemos definir um tipo especial de set. Vamos definir como \textbf{conjunto vazio} ou \textbf{empty set} um conjunto que possua o cardinal igual a zero. Usaremos o token "$\emptyset$" para definir a relação abaixo:

\begin{center}
	$|\emptyset| = 0$
\end{center}


%%%%%%%%%%%%%%%%%%%%%%%%%%%%%%%%%%%%%%%%%%%%%%%%%%%
%                 CHAPTER                         %
%%%%%%%%%%%%%%%%%%%%%%%%%%%%%%%%%%%%%%%%%%%%%%%%%%%
\chapter{Lógica}


%%%%%%%%%%%%%%%%%%%%%%%%%%%%%%%%%%%%%%%%%%%%%%%%%%%
%                 CHAPTER                         %
%%%%%%%%%%%%%%%%%%%%%%%%%%%%%%%%%%%%%%%%%%%%%%%%%%%
\chapter{Contagem}
\chapter{Prova Direta}
\chapter{Prova Contra-positiva}
\chapter{Prova por Contradição}
\chapter{Prova de Afirmações Não-Condicionais}
\chapter{Provas Envolvendo Conjuntos}
\chapter{Contraprova}
\chapter{Indução Matemática}
\chapter{Relações}
\chapter{Funções}
\chapter{Provas com Calculus}
\chapter{Cardinalidade de Conjuntos}

\end{document}