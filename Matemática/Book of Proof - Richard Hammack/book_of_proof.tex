%%%%%%%%%%%%%%%%%%%%%%%%%%%%%%%%%%%%%%%%%%%%%%%%%%%
%% LaTeX book template                           %%
%% Author:  Amber Jain (http://amberj.devio.us/) %%
%% License: ISC license                          %%
%%%%%%%%%%%%%%%%%%%%%%%%%%%%%%%%%%%%%%%%%%%%%%%%%%%

\documentclass[a4paper,11pt]{book}

\usepackage[T1]{fontenc}
\usepackage[utf8]{inputenc}
\usepackage{lmodern}
\usepackage{hyperref}
\usepackage{graphicx}
\usepackage[portuguese]{babel}
\usepackage{amsfonts}
\usepackage[usenames,dvipsnames]{xcolor}
\usepackage{mathtools}
\usepackage{amssymb} % alguns simbolos matematicos
\usepackage{mathrsfs} % letras cursivas

% para usar definições
\usepackage{amsthm}

\theoremstyle{definition}

\newtheoremstyle{break}% name
  {10pt}%         Space above, empty = `usual value'
  {10pt}%         Space below
  {}% Body font
  {}%         Indent amount (empty = no indent, \parindent = para indent)
  {\bfseries}% Thm head font
  {}%        Punctuation after thm head
  {\newline}% Space after thm head: \newline = linebreak
  {}%         Thm head spec

\theoremstyle{break}

\newtheorem{definition}{Definição}[section]
\newtheorem{fact}{Fato}[section]
\newtheorem{demonstration}{Demonstração}[section]

% tirando identação dos paragrafos
\setlength{\parindent}{0ex}

% coding examples in latex file
\usepackage{listings}
\usepackage{xcolor}

\definecolor{codegreen}{rgb}{0,0.6,0}
\definecolor{codegray}{rgb}{0.5,0.5,0.5}
\definecolor{codepurple}{rgb}{0.58,0,0.82}
\definecolor{backcolour}{rgb}{0.95,0.95,0.92}

\lstdefinestyle{mystyle}{
    backgroundcolor=\color{backcolour},   
    commentstyle=\color{codegreen},
    keywordstyle=\color{magenta},
    numberstyle=\tiny\color{codegray},
    stringstyle=\color{codepurple},
    basicstyle=\ttfamily\footnotesize,
    breakatwhitespace=false,         
    breaklines=true,                 
    captionpos=b,                    
    keepspaces=true,                 
    numbers=left,                    
    numbersep=5pt,                  
    showspaces=false,                
    showstringspaces=false,
    showtabs=false,                  
    tabsize=2
}

\lstset{style=mystyle}

%%%%%%%%%%%%%%%%%%%%%%%%%%%%%%%%%%%%%%%%%%%%%%%%%%%%%%%%%%%%%%%%%%%%%%%%%%%%%%%%
% 'dedication' environment: To add a dedication paragraph at the start of book %
% Source: http://www.tug.org/pipermail/texhax/2010-June/015184.html            %
%%%%%%%%%%%%%%%%%%%%%%%%%%%%%%%%%%%%%%%%%%%%%%%%%%%%%%%%%%%%%%%%%%%%%%%%%%%%%%%%
\newenvironment{dedication}
{
   \cleardoublepage
   \thispagestyle{empty}
   \vspace*{\stretch{1}}
   \hfill\begin{minipage}[t]{0.66\textwidth}
   \raggedright
}
{
   \end{minipage}
   \vspace*{\stretch{3}}
   \clearpage
}

%%%%%%%%%%%%%%%%%%%%%%%%%%%%%%%%%%%%%%%%%%%%%%%%
% Chapter quote at the start of chapter        %
% Source: http://tex.stackexchange.com/a/53380 %
%%%%%%%%%%%%%%%%%%%%%%%%%%%%%%%%%%%%%%%%%%%%%%%%
\makeatletter
\renewcommand{\@chapapp}{}% Not necessary...
\newenvironment{chapquote}[2][2em]
  {\setlength{\@tempdima}{#1}%
   \def\chapquote@author{#2}%
   \parshape 1 \@tempdima \dimexpr\textwidth-2\@tempdima\relax%
   \itshape}
  {\par\normalfont\hfill--\ \chapquote@author\hspace*{\@tempdima}\par\bigskip}
\makeatother


%%%%%%%%%%%%%%%%%%%%%%%%%%%%%%%%%%%%%%%%%%%%%%%%%%%
% First page of book which contains 'stuff' like: %
%  - Book title, subtitle                         %
%  - Book author name                             %
%%%%%%%%%%%%%%%%%%%%%%%%%%%%%%%%%%%%%%%%%%%%%%%%%%%
% Book's title and subtitle
\title{\Huge \textbf{Book of Proof} \\ 
\Large Third Edition \\
\huge by Richard Hammack}

% Author
\author{\textsc{Resumo por Bruno de M. Ruas}}


\begin{document}

\frontmatter
\maketitle

\tableofcontents
%\listoffigures
%\listoftables

\mainmatter

%%%%%%%%%%%%%%%%%%%%%%%%%%%%%%%%%%%%%%%%%%%%%%%%%%%
%                    PART                         %
%%%%%%%%%%%%%%%%%%%%%%%%%%%%%%%%%%%%%%%%%%%%%%%%%%%
\part{Fundamentos}

%%%%%%%%%%%%%%%%%%%%%%%%%%%%%%%%%%%%%%%%%%%%%%%%%%%
%                 CHAPTER                         %
%%%%%%%%%%%%%%%%%%%%%%%%%%%%%%%%%%%%%%%%%%%%%%%%%%%
\chapter{Conjuntos}

\begin{chapquote}{página 3}
	``The theory of sets is a language that is perfectly suited to describing and explaning all types of mathematical structures.''
\end{chapquote}


\section*{Aviso ao Leitor}
Bem-vindo ao início de uma jornada consideravelmente longa. Esse texto é um resumo (mais conciso e menos didático) do livro do professor Hammack. O objetivo desse presente manual é servir como material de revisão e auxílio aos que quiserem seguir o caminho proposto no \textbf{Projeto Matemática} do site \href{https://economiamainstream.com.br/artigo/matematica/}{\textbf{Economia Mainstream}}. A leitura do material original é fortemente indicada e encorajada por parte dos que elaboraram o presente manual. Os exercícios contidos no livro, por outro lado, são obrigatórios. Você deve tentar resolver o máximo possível. Quaisquer dúvidas podem ser enviadas nos comentários do projeto no site citado acima.

\section{Introdução}
Um \textbf{conjunto} (set)\footnote{Eu vou intercalar bastante o uso dos termos em português e inglês.} é uma lista de \textbf{elementos}. Normalmente denotados por uma letra maiúscula. Por exemplo:

\begin{center}
	$ A = \{1 , 2 , 3 , 4 , ... \} $
\end{center}

\textbf{Regra}: Dois sets $A$ e $B$ são \textbf{iguais} se possuírem exatamente os mesmos elementos. Não importando a ordem desses elementos dentro de cada set. Ou seja, $ \{ 1,2,3 \} = \{3,1,2\}$.
\\
\\
Vamos definir um símbolo para sinalizar se um determinado elemento $(x)$ pertence ou não a um determinado set qualquer $(A)$. Para tal relação usaremos o símbolo $``\in"$ se $x$ for um elemento de $A$ ou, caso contrário, usaremos $``\notin"$ se $x$ não for um elemento de $A$.
\\
\\
É provável que, em algum momento, seja necessário contar a quantidade de elementos em um dado set qualquer $A$. Chamaremos essa relação de \textbf{cardinalidade} ou \textbf{tamanho} do set $A$. O símbolo usado será duas barras em volta do set do seguinte modo: $``|A|"$.
\\
\\
A partir dessas duas relações já podemos definir um tipo especial de set. Vamos definir como \textbf{conjunto vazio} ou \textbf{empty set} um conjunto que possua o cardinal igual a zero. Usaremos o símbolo $`` \ \emptyset \ "$ para definir a relação a seguir:

\begin{center}
	$|\emptyset| = 0$
\end{center}

Em várias situações não vale a pena construir sets apenas com uma lista-exemplo de alguns dos seus elementos. Imagine um set de todos os números pares, por exemplo, ou um set de todos os números que começam com $3$ e terminam com $4$ ou qualquer outra regra mais específica. Para essas situações usamos a \textbf{notação de formação de conjuntos (set builder notation)}. Como no exemplo abaixo:

\begin{center}
	$ E = $ \textcolor{red}{$\{$} \textcolor{blue}{$2n$} \textcolor{OliveGreen}{$:$} \textcolor{Brown}{$n$} \textcolor{Orange}{$\in$} $\mathbb{Z} \} $ \footnote{Aprender a ler a notação de formação de conjuntos é, basicamente, aprender a ler Matemática. Se esforce para entender sempre que usarmos essa técnica na hora de definir conjuntos. Para ajudar, eu vou colocar no rodapé a leitura em alguns casos.}
\end{center}

A matemática é uma linguagem que consegue dizer muita coisa com poucos símbolos. Ao longo desse curso, você será capaz de ler esses símbolos e compreender corretamente o que o autor quis dizer por meio deles. Para facilitar essa primeira leitura, eu colori cada símbolo da expressão acima com a cor correspondente da passagem a seguir. Perceba como um pequeno símbolo pode significar bastante coisa. A leitura da expressão acima é: "O conjunto $E$ é igual ao \textcolor{red}{conjunto dos elementos da forma} \textcolor{blue}{$2n$} \textcolor{OliveGreen}{tal que} \textcolor{Brown}{$n$} \textcolor{Orange}{é um elemento do conjunto} $\mathbb{Z}$".
\\
\\
Podemos resumir essa notação de formação de conjuntos como "Conjunto = {Expressão : Regra}". É bem comum vermos notações onde os dois pontos são trocados por uma barra: "Conjunto = {Expressão | Regra}". Nesse livro o autor preferiu a notação com dois pontos.
\\
\\
Existem alguns conjuntos que são famosos ao ponto de terem nomes e símbolos próprios.
\begin{itemize}
	\item[] $\emptyset = \{ \}$. Conjunto Vazio
	\item[] $\mathbb{N} = \{ 1, 2, 3, 4, ... \}$\footnote{Perceba que, para o autor do livro, $0 \notin \mathbb{N}$. Usaremos $\mathbb{N}^0$ para expressar os naturais com o elemento 0.\href{https://www.youtube.com/watch?v=j9WmXliT0lM}{ Link para um video bacana sobre o assunto.}}. Conjunto dos Naturais
	\item[] $\mathbb{Z} = \{ ..., -2, -1, 0, 1, 2, ...  \}$. Conjunto dos Inteiros
	\item[] $\mathbb{Q} = \{ x : x = m/n, \ onde \ m,n \in \mathbb{Z} \  e \  n \neq 0 \}$. Conjunto dos Racionais\footnote{Lê-se: O set dos Racionais é igual ao conjunto dos elementos $x$ tal que $x$ é formado pela divisão de $m$ por $n$, de modo que $m$ e $n$ são número inteiros e $n$ não é zero.}
	\item[] \includegraphics[scale=0.11]{images/real_line.png}. A Reta Real
\end{itemize}

Como o conjunto dos número reais pode ser descrito como pontos em uma reta numérica infinita. Se tivermos dois pontos quaisquer $a$ e $b$, de modo que $a , b \in \mathbb{R}$ e $a < b$. Temos infinitos elementos entre esses dois pontos. Por causa dessa propriedade, teremos que usar um novo símbolo para se referir aos conjuntos que são melhor descritos em termos de \textbf{intervalos} entre pontos. Abaixo coloquei uma coluna com uma representação gráfica e, ao lado, uma coluna com a respectiva definição por set builder notation.
\begin{center}
	\includegraphics[scale=0.7]{images/intervals.png}
\end{center}

\section{Produto Cartesiano}
\begin{definition}[Par Ordenado]
Um \textbf{par ordenado} é uma lista\footnote{Nós definiremos uma lista no cap. 03} na forma $(x, y)$ que contém dois elementos (nesse caso, um $x$ e um $y$). Onde esses dois elementos ficam entre parênteses e separados por uma vírgula.
\end{definition}

\textbf{Regra}: Diferente dos conjuntos, a ordem dos elementos dos pares ordenados importa. Ou seja, $(x,y) \neq (y,x)$. 
\\
\\
Agora que temos a definição de par ordenado. Podemos escrever conjuntos usando esse novo conceito.

\begin{definition}[Produto Cartesiano]
O \textbf{produto cartesiano} de dois sets $A$ e $B$ é um outro set cujo símbolo é $``A \times B"$ e é definido como:
\begin{center}
$A \times B = \{ (a,b) : a \in A, b \in B \}$\footnote{Lê-se: O produto cartesiano dos conjuntos $A$ e $B$ é formado pelos pares ordenados $(a,b)$ de modo que as primeiras coordenadas são elementos de $A$ e as segundas são elementos de $B$.}
\end{center}
\end{definition}

Perceba que, se $A$ e $B$ são finitos, então $| A \times B | = |A| \ . \ |B|$. Isso é, o cardinal do produto cartesiano de dois sets é igual à multiplicação dos cardinais dos dois conjuntos.\footnote{Ao longo da etapa de Fundamentos do Projeto Matemática, faremos várias afirmações sem a devida demonstração mas, a medida que você entrar no curso de Análise, não faremos afirmações sem as devidas provas.}
\\
\\
Podemos construir um produto cartesiano onde os conjuntos $A$ e $B$ são iguais. Por exemplo: $\mathbb{R} \times \mathbb{R} = \{ (x,y) : x,y \in \mathbb{R} \}$.
\\
\\
Podemos expandir o par ordenado para infinitos elementos. Para isso vamos criar um novo conceito mais geral que abarcará o conceito de par ordenado. Chamaremos de \textbf{n-upla} a coordenada de $n$ elementos do modo $(x_1, x_2, \dots, x_n)$. Nesse conceito mais geral, um par ordenado nada mais é que uma n-upla cujo $n = 2$.
\\
\\
Para simplificar essa expressão onde temos um produto cartesiano de sets iguais, vamos criar um novo conceito que chamaremos de \textbf{potência cartesiana (Cartersian power)}. Desse modo, podemos definir o exemplo de $``\mathbb{R} \times \mathbb{R}"$ como simplesmente $``\mathbb{R}^2"$. Mais genericamente, dizemos que, para qualquer set $A$ e um $n$ positivo, o cartesian power $A^n$ será definir como:

\begin{center}
	$ A^n = \underbrace{A \times A \times ... \times A}_\text{n \ vezes} = \{ (x_1,x_2, ... , x_n) : x_1,x_2, ... , x_n \in A \} $ \footnote{Lê-se: A potência cartesiana $A^n$ é igual ao conjunto das n-uplas cujas $n$ coordenadas são elementos do conjunto $A$.}
\end{center}

\section{Subconjuntos}
Nós já aprendemos a relacionar elementos e conjuntos mas agora vamos definir um método de relacionar conjuntos entre si. A primeira relação que vamos explorar é a que expressa a situação onde todos os elementos de um conjunto também são elementos de outro conjunto.

\begin{definition}[Subconjunto]
Suponha que existam dois sets $A$ e $B$. Se todos os elementos de $A$ também forem elementos de $B$, dizemos que $A$ é um \textbf{subconjunto (subset)} de $B$. O símbolo usado para expressar essa relação é $``\subseteq"$, ou seja, $A \subseteq B$ quer dizer que $A$ é subconjunto de $B$. Caso exista um elemento de $A$ que não seja um elemento de $B$, então escrevemos que $A \nsubseteq B$.
\end{definition}

\textbf{Atenção 1}: Uma consequência direta dessa definição de subconjunto é o fato que $\emptyset \subseteq B$ para qualquer conjunto $``B"$. A demonstração dessa afirmação é simples: Suponha que exista algum conjunto $Z$ onde $\emptyset \nsubseteq Z$. Isso significaria que existe algum $x \in \emptyset$ que não é um elemento de $Z$, ou seja, $x \notin Z$. Mas, por definição, $x \notin \emptyset$, desse modo, $\emptyset \subseteq Z$.
\\
\\
\textbf{Atenção 2}: É trivial o fato que dado um conjunto qualquer $A$, todos os elementos de $A$ pertencem a ele mesmo. Isso implica que $A \subseteq A$. Portanto, todo conjunto é subconjunto de si mesmo.
\\
\\
\textbf{Atenção 3}: Como vimos antes: $\emptyset \subseteq B$, para qualquer set B. Acontece que também é verdadeiro o fato que $\emptyset \subseteq \emptyset$. Uma vez que, se $\emptyset \nsubseteq \emptyset$ existiria algum $x$ de modo que $x \in \emptyset$ e $x \notin \emptyset$. O que é uma clara contradição.

\section{Conjunto de Partes}
\begin{definition}[Conjunto de Partes]
Dado um set qualquer $B$, o seu \textbf{conjunto de partes} ou \textbf{power set} será outro set escrito como $\mathscr{P}(B)$ e definido como:
\begin{center}
	$\mathscr{P}(B) = \{ X : X \subseteq B \}$
\end{center}
\end{definition}

\textbf{Dica}: Tem bastante informação interessante no livro. Como esse aqui é só um resumo, não vou entrar muito além da definição. Mas recomendo a leitura do material original.

\section{União, Intersecção e Diferença}
Já vimos como podemos relacionar conjuntos por \textbf{produto cartesiano} para gerar outros conjuntos. Agora vamos expandir ainda mais nosso ferramental de operações entre conjuntos.

\begin{definition}[União]
Dados dois conjuntos $F$ e $G$. A \textbf{união} entre eles será um novo set denotado por $``F \cup G"$ e definido como:
\begin{center}
	$F \cup G = \{ x : x \in F \ ou \ x \in G \}$
\end{center}
\end{definition}

\begin{definition}[Intersecção]
Dados dois conjuntos $F$ e $G$. A \textbf{intersecção} entre eles será um novo set denotado por $``F \cap G"$ e definido como:
\begin{center}
	$F \cap G = \{ x : x \in F \ e \ x \in G \}$
\end{center}
\end{definition}

\begin{definition}[Diferença]
Dados dois conjuntos $F$ e $G$. A \textbf{diferença} entre eles será um novo set denotado por $``F - G"$ e definido como:
\begin{center}
	$F - G = \{ x : x \in F \ e \ x \notin G \}$
\end{center}
\end{definition}

\textbf{Dica}: Esses conceitos são muito importantes. Mas pro curso não ficar muito grande, vou me manter só nos conceitos também. Vá ler o material original caso tenha dificuldade.

\section{Complemento}
Quando lidamos com conjuntos é comum supor que há um conjunto maior que contém todos os outros. A esse set geral chamamos de \textbf{conjunto universo} ou \textbf{conjunto universal}.

\begin{definition}[Complemento]
Dado um conjunto qualquer $H$ e o seu conjunto universo $U$. O \textbf{complemento} de $H$ é um novo set denotado por $``\overline{H}"$ e definido por:
\begin{center}
	$\overline{H} = U - H$
\end{center}
\end{definition}

\section{Diagramas de Venn}

Essa sessão eu pulei integralmente. Diagramas de Venn são ótimos pra se ter uma intuição sobre todos os conceitos que vimos até agora. Mas não são usados para provas matemáticas. Ainda vale a leitura do capítulo.

\section{Conjuntos Indexados}
As vezes é necessário trabalhar com uma quantidade consideravelmente grande de conjuntos. Para esses casos, usamos uma técnica de simplificação que é adicionar um índice numérico subscrito à alguma letra maiúscula. Desse modo, ao invés de trabalharmos com sets $A$, $B$ e $C$, podemos trabalhar com os sets $A_1$, $A_2$ e $A_3$.
\\
\\
Podemos relacionar esses subscritos à um outro set. O nome dado a esse set é \textbf{conjunto índice (index set)}. Nos exemplos acima, podemos dizer que todos os subscritos pertencem ao conjunto $\{ 1 , 2 , 3 \}$.
\\
\\
Agora podemos adicionar essa técnica às relações de união e intersecção entre esses conjuntos indexados para um numero arbitrariamente grande. Além disso, vamos usar uma notação similar a do somatório\footnote{\href{https://pt.khanacademy.org/math/algebra2/sequences-and-series/alg2-sigma-notation/v/sigma-notation-sum}{Link para uma aula sobre a notação Sigma.}} para definir essas relações.

\begin{definition}[União de Sets Indexados]
Dados os conjuntos $A_1$, $A_2$, $A_3$, $\dots$, $A_n$ e o index set $I = \{ 1 , 2 , \dots , n \}$, temos que
\begin{center}
	$\bigcup_{i \in I}\limits = \{ x : x \in A_i \ para \ algum \ A_i, \ onde \  i \in I \}$
\end{center}
\end{definition}

\begin{definition}[Intersecção de Sets Indexados]
Dados os conjuntos $A_1$, $A_2$, $A_3$, $\dots$, $A_n$ e o index set $I = \{ 1 , 2 , \dots , n \}$, temos que
\begin{center}
	$\bigcap_{i \in I}\limits = \{ x : x \in A_i \ para \ todo \ A_i, \ onde \  i \in I \}$
\end{center}
\end{definition}

O livro tem dois exemplos bem interessantes da aplicação dos conceitos que acabamos de definir. A essa altura você deve conseguir entender os dois.

\section{Conjuntos que são Sistemas Numéricos}

A maioria dos conjuntos que trabalhamos são conjuntos que possuem estruturas e propriedades especiais. No caso dos conjuntos numéricos tomamos como certo que os seus elementos podem ser somados, multiplicados e possuem relações que obedecem as regras que passamos todo o ensino infantil, fundamental e médio aprendendo e aplicando. Como esse livro é introdutório, todas essas características clássicas dos sistemas numéricos serão tomadas como verdade. Mas saiba que as relações que achamos ser naturais possuem comprovações bastante complexas que você pode procurar por conta própria.
\\
\\
Aqui o autor elenca algumas propriedades que tomaremos como verdades sem que sejam devidamente definidas e demonstradas:
\begin{itemize}
	\item Propriedade Comutativa/Associativa/Distributiva da Adição/Subtração/Multiplicação/Divisão
	\item Ordenação Natural dos elementos numéricos de $\mathbb{R}$
	\item Os subconjuntos de $\mathbb{N}$ obedecem ao \href{https://pt.wikipedia.org/wiki/Princ%C3%ADpio_da_boa_ordena%C3%A7%C3%A3o}{Princípio da Boa Ordenação}.
\end{itemize}

\textbf{Comentário}: Agora a gente vai adentrar um pouco nas propriedades que podemos derivar desses pressupostos acima. Pode parecer que é um papo chato, mas a sua missão é se certificar que você é capaz de compreender toda a explicação. Vença a preguiça.
\\
\\
Uma conclusão que podemos tirar do princípio da boa ordenação é que dado um conjunto não nulo qualquer $A \subseteq \mathbb{N}$ sempre vai haver um $x_0 \in A$ que seja o seu \textbf{menor elemento}. De modo parecido, para qualquer $b \in \mathbb{Z}$, qualquer conjunto não nulo $A \subseteq \{ b, b+1, b+2, b+3, \dots \}$ também possui um \textbf{menor elemento}.

\begin{fact}[Division Algorithm]
Dados dois inteiros $a$ e $b$, onde $b > 0$. Existem outros dois inteiros únicos $q$ e $r$ para qual $a = qb + r$ e $0 \leq r < b$.
\end{fact}

Agora o autor demonstra a existência de $r$ e $q$ com suas propriedades. Talvez depois eu estenda a prova para a unicidade\footnote{Eu achei uma demonstração em JOHNSTON, William; MCALLISTER, Alex. A transition to advanced mathematics: a survey course. OUP USA, 2009. p. 118.} desses valores\footnote{Passa lá no Economia Mainstream e me dá uma cobrada nos comentários}.

\begin{demonstration}[Division Algorithm - Existência de $r$ e $q$]
Dados $a,b \in \mathbb{Z}$ e $b > 0$, é possível criar um set do tipo:
\begin{center}
	$ A = \{ a - xb : x \in \mathbb{Z}, 0 \leq a - xb \} \subseteq \mathbb{N}^0 $
\end{center}
Desse modo, $A \subseteq \mathbb{N}^0$. Por causa disso, podemos aplicar o princípio da boa ordenação em $A$ e dizer que existe algum elemento $r$ que seja o menor elemento de $A$.
\\
\\
Como $r \in A$ ele pode ser escrito da forma $r = a - qb$ onde $x = q \in \mathbb{Z}$. Desse fato podemos tirar duas informações úteis: 1) $a = r + qb$ e 2) Como $r \in \mathbb{N}^0$, então $r \geq 0$.
\\
\\
Agora só nos resta provar que $r < b$ mas para isso, vamos usar o pensamento contrário: O que aconteceria se $r \geq b$?
\\
\\
Ora, se $r \geq b$, então a subtração $r - b$ será um número positivo, portanto será também um elemento de $\mathbb{N}^0$. Podemos reescrever essa subtração como $r - b = (a - qb) - b = a - (q + 1)b$. 
\\
\\
Mas veja só que estranho, $q + 1$ certamente será um elemento de $\mathbb{Z}$. Logo, o número expresso por $a - (q + 1)b$ também será um elemento de $A$ (nesse caso, $(q + 1)$ é o $x$). Portanto, não é possível que $r$ seja o seu menor elemento visto que $r - b$ também é um elemento de $A$ e é menor que $r$. Isso é uma clara contradição. Isso é justamente isso que queríamos mostrar: Quando tomamos $r \geq b$ acabamos com uma contradição\footnote{O nome dessa técnica é prova por contradição. Falaremos dela mais pra frente no curso.}, portanto, só podemos aceitar o oposto, ou seja, sabemos que $r < b$. Com isso finalizamos a demonstração da existência de $r$ e $q$ com as propriedades anteriormente citadas $\blacksquare$\footnote{Esse quadrado preto é usado para pontuar o final de uma demonstração.}
\end{demonstration}


\section{O Paradoxo de Russell}

Até agora trabalhos a distinção entre "elementos" e "conjuntos". Mas, na verdade, qualquer número (ou seja, elemento de sets numéricos) pode, sim, ser interpretado como um conjunto.\footnote{Existe um mundo de teoria sobre conjuntos, nós não temos tempo pra entrar muito fundo nessa questão. Então vá atrás de livros sobre teoria dos conjuntos e seja feliz.}. Até mesmos as operações matemáticas podem ser definidas usando-se teoria dos conjuntos. O Autor vai até mais longe "Qualquer entidade matemática é um conjunto, mesmo que não escolhamos pensar desse modo".
\\
\\
Essa parte do paradoxo de Russell não serve pro resto do livro mas é bem legal de saber. O que Bertrand Russell propôs foi o seguinte conjunto:
\begin{center}
	$ A = \{ X : X $ é um set e $X \notin X \}$\footnote{Ou seja, A é formado por todos os conjuntos que não possuem a si mesmo como elemento.}
\end{center}
E perguntou: "O conjunto $A$ é um elemento de si mesmo?".
\\
\\
Acredite, apenas esse enunciado foi uma baita dor de cabeça para os matemáticos na época. Esse assunto não é necessário para a compreensão do resto desse material. Se você quiser entender um pouco mais, confira esse \href{https://www.youtube.com/watch?v=AQTTYAM8BF0}{link} ou esse \href{https://www.youtube.com/watch?v=0Bs0lJRxOaI}{link}, ou ainda esse último \href{https://www.youtube.com/watch?v=HeQX2HjkcNo}{link}.
\\
\\
\textbf{Dica}: Vai ler o livro nessa parte. O professor explica bem melhor sobre o paradoxo.

%%%%%%%%%%%%%%%%%%%%%%%%%%%%%%%%%%%%%%%%%%%%%%%%%%%
%                 CHAPTER                         %
%%%%%%%%%%%%%%%%%%%%%%%%%%%%%%%%%%%%%%%%%%%%%%%%%%%
\chapter{Lógica}

\begin{chapquote}{página 34}
	``Logic is a systematic way of thinking that allow us to parse the meanings of sentences and to deduce new information from old information. [...] Logic is  a process of deducing information correctly, not just deducing correct information''
\end{chapquote}

\section{Proposições}

O estudo da lógica começa com as proposições. Uma \textbf{proposição} é uma sentença ou expressão matemática que seja definitivamente verdadeira ou falsa. Em cima dessas proposições é que aplicamos a lógica para produção de novas proposições. Qualquer resultado ou teorema provado é, na verdade, uma proposição\footnote{Isso inclui o teorema de Pitágoras, por exemplo.}. 
\\
\\
\textbf{Aviso}: Ao longo desse capítulo a palavra "proposição" \ vai ser repetida muitas vezes. Essa repetição é proposital. Eu quero que você internalize a importância desse conceito para a análise matemática.
\\
\\
Podemos usar letras para nomear proposições. Por exemplo, a proposição "Se um número $x$ é múltiplo de 6, então $x$ é par" \  pode ser resumida em uma letra. Nesse caso diremos que "$P$" \ é a proposição acima. Dessa feita, sempre que dissermos $P$ é falso ou $P$ é verdadeiro, estamos nos referindo a proposição completa.
\\
\\
Quando uma proposição possuir variáveis, podemos escreve-la de uma maneira similar às funções. No exemplo do parágrafo acima, podemos usar o símbolo $P(x)$ para expressar que a proposição $P$ possui uma variável $x$ dentro dela.\footnote{Fique atento para não confundir quando estivermos nos referindo a funções e proposições.}
\\
\\
Uma \textbf{sentença aberta} é uma proposição cuja validade depende da sua variável. Ou seja, dependendo do valor que você der para a variável, a proposição pode ser verdadeira ou falsa.
\\
\\
Esse livro do professor Hammack é justamente sobre o método de como se provar que uma proposição é verdade ou falsa. Existem estratégias de raciocínio que podem ser usadas para demonstrar proposições e, consequentemente, teoremas complexos. Nosso objetivo nesse curso é lhe ensinar essas técnicas.

\section{Operadores: e, ou, não}

Até agora foi dito que você pode usar proposições junto à lógica para se chegar a outras proposições. Então vamos aprender como trabalhar com mais de uma proposição por meio dos \textbf{operadores lógicos}.
\\
\\
O primeiro operador que vamos aprender é o "$e$". Dadas as proposições $P$ e $Q$. Podemos criar uma nova proposição através desse operador. Quando escrevemos $R: P \ e \ Q$, estamos criando uma nova proposição $R$ que é composta de das duas proposições $P$ e $Q$. O símbolo usado para esse conectivo é o "$\land$". Aqui em baixo vamos colocar a tabela verdade\footnote{Se você não sabe o que são tabelas-verdades então dá uma lida nesse \href{https://www.youtube.com/watch?v=hWEZsyF3ZZc}{link}.} desse operador\footnote{"V" \ significa Verdadeiro e "F" \  significa Falso.}.

\begin{center}
\begin{tabular}{ c c || c }
 P & Q & $P \land Q$ \\ 
 \hline
 V & V & V \\  
 V & F & F \\  
 F & V & F \\  
 F & F & F
\end{tabular}
\end{center}

Outro operador lógico clássico é o "ou". O símbolo dele é o "$\lor$". Eu sei, parece muito com o símbolo do "e" \ mas a  gente não pode fazer nada a não ser decorar bem a distinção entre esses dois símbolos. A tabela-verdade do "ou" \ segue abaixo.

\begin{center}
\begin{tabular}{ c c || c }
 P & Q & $P \lor Q$ \\ 
 \hline
 V & V & V \\  
 V & F & V \\  
 F & V & V \\  
 F & F & F
\end{tabular}
\end{center}

\textbf{Atenção}: O significado da palavra "ou" \ na linguagem do dia a dia é diferente do significado empregado no contexto da lógica. Quando usamos o conectivo "ou" \ na lógica, fica implícita a possibilidade de ambas as proposições sejam verdadeiras ao mesmo tempo. Para as situações onde queremos que apenas uma proposição seja aceita mas não ambas, usamos um "ou exclusivo"\footnote{O símbolo desse é o "$\oplus$".} \ dos seguintes modos: 

\begin{itemize}
\item ou $P$ ou $Q$
\item $P$ ou $Q$, mas não ambos
\item Exatamente um de $P$ ou $Q$
\end{itemize}

O último operador que veremos é a \textbf{negação}. Ela inverte a condição da proposição onde é aplicada. Seu símbolo é o "$\sim$"\footnote{Existem outras representações desse operador: $\lnot$ , $\veebar$ , $\dot\lor$}, ou seja, se uma proposição é verdadeira, ao aplicarmos a negação à ela, tornamos essa proposição falsa. A tabela-verdade desse operador é bem simples.

\begin{center}
\begin{tabular}{ c || c }
 P & $\sim P$ \\ 
 \hline
 V & F \\  
 F & V
\end{tabular}
\end{center}

\section{Proposições Condicionais}

Podemos ainda combinar proposições de outra maneira. Sejam as proposições $P$ e $Q$. Podemos dizer que "Se $P$ for verdade, então $Q$ também será verdadeiro". O símbolo usado para esse conectivo se...então é o "$\implies$". Uma proposição desse tipo é chamada de \textbf{proposição condicional}. Leia a página 43 do livro caso você esteja com dúvida sobre as últimas duas linhas da tabela-verdade.\footnote{O autor dá a dica de você pensar nas proposições compostas como promessas feita por alguém. Isso ajuda muito a entender como as tabelas-verdade são construídas.}

\begin{center}
\begin{tabular}{ c c || c }
 P & Q & $P \implies Q$ \\ 
 \hline
 V & V & V \\  
 V & F & F \\  
 F & V & V \\  
 F & F & V
\end{tabular}
\end{center}

\textbf{Dica}: Nessa parte do livro o professor coloca bastante esforço em explicar esse conceito. Se você não conseguir entender essa tabela-verdade acima, recomendo dar uma lida seção que começa na página 42.

\section{Proposições Bicondicionais}

Existem situações onde duas proposições são mutuamente condicionais, ou seja, $P \implies Q$ e $Q \implies P$. Nesses casos usamos o conectivo "se e somente se" cujo símbolo é "$\iff$".  A tabela-verdade segue abaixo.

\begin{center}
\begin{tabular}{ c c || c }
 P & Q & $P \iff Q$ \\ 
 \hline
 V & V & V \\  
 V & F & F \\  
 F & V & F \\  
 F & F & V
\end{tabular}
\end{center}

\textbf{Desafio}: Na prática, a proposição bicondicional $P \iff Q$ nada mais é do que a proposição composta $(P \implies Q) \land (Q \implies P)$. Leia a seção 2.5 e volte aqui. Seu desafio é provar essa equivalência.\footnote{Posta uma foto da tabela-verdade composta e me marca no  \href{https://twitter.com/bruno_ruas2}{twitter}.}.


\section{Tabelas-Verdade para Proposições}

Agora que você sabe as tabelas-verdade de $\land$,$\lor$,$\sim$,$\implies$ e $\iff$, você deve internalizá-las de modo a serem muito naturais ao avaliar a validade de uma proposição cuja construção utilize esses operadores.
\\
\\
Vamos trabalhar uma proposição composta que expressa a seguinte situação: Dadas duas proposições $P$ e $Q$, somente uma delas é verdadeira mas não ambas. Não existe um operador que nos permita construir uma proposição apenas com um único símbolo. Para podermos construir tal proposição temos que combinar nossos operadores do seguinte modo:

\begin{center}
	$(P \lor Q) \ \land \ \sim (P \land Q)$
\end{center}

Essa proposição pode assustar numa primeira olhada, contudo, ela expressa exatamente nossa intenção anteriormente explicitada. Podemos resumi-la como: "$P$ ou $Q$ são verdadeiras, mas não é o caso onde $P$ e $Q$ são verdadeiras.
\\
\\
A tabela-verdade dessa proposição é a seguinte:

\begin{center}
\begin{tabular}{c c || c c || c || c}
 P & Q & $(P \lor Q)$ & $(P \land Q)$ & $\sim (P \land Q)$ & $(P \lor Q) \ \land \ \sim (P \land Q)$\\
 \hline
 V & V & V & V & F & F \\  
 V & F & V & F & V & V \\  
 F & V & V & F & V & V \\  
 F & F & F & F & V & F
\end{tabular}
\end{center}

Calma. É bem mais tranquilo do que parece. Vamos analisar essa tabela por partes. Nas primeiras duas colunas temos as proposições iniciais $P$ e $Q$. Nas linhas estão as situações onde testamos o que acontece quando elas são verdadeiras ou falsas. Na segunda parte temos as proposições compostas por $\lor$ e $\land$ (que você já conhece a tabela-verdade). Na parte 3 temos apenas a negação da coluna $(P \land Q)$. Se criarmos mais duas proposições do tipo $Z : (P \lor Q)$ e $W : \  \sim(P \land Q)$. Podemos reduzir a última coluna como a tabela-verdade abaixo:

\begin{center}
\begin{tabular}{ c c || c c || c }
 P & Q & Z & W & $Z \land W$ \\ 
 \hline
 V & V & V & F & F \\  
 V & F & V & V & V \\  
 F & V & V & V & V \\  
 F & F & F & V & F
\end{tabular}
\end{center}

Que é exatamente igual à última coluna da tabela-verdade anterior. Portanto, nossa proposição é verdadeira exatamente quando $P$ ou $Q$ são verdadeiros mas não quando ambos são verdadeiros. Parabéns por ter entendido até aqui!
\\
\\
\textbf{Comentário}: Eu peguei esse exemplo direto do livro. Contudo, eu pensei um pouco e uma equivalência a essa proposição composta (bem mais simples) seria $\sim(P \iff Q)$\footnote{Você consegue ver isso?}.

\section{Equivalência Lógica}

Nesse meu comentário acima já vemos um exemplo de equivalência lógica. Quando temos tabelas-verdade iguais para proposições diferentes. Podemos dizer que elas são \textbf{logicamente equivalentes}.
\\
\\
Esse conceito é importante porque podemos trabalhar o mesmo problema de diferentes abordagens e acabarmos chegando no mesmo resultado. Cada abordagem pode ter facilidades que queiramos explorar e que vão nos permitindo desenrolar o pensamento necessário para uma demonstração mais complexa.
\\
\\
Muitos teoremas matemáticos usam a forma $P \implies Q$. Mas, não raramente, precisamos usar a equivalência lógica e trabalhar com a proposição $\sim (Q) \implies \sim (P)$. Para ver essa equivalência, basta construir as tabelas-verdade das duas proposições.\footnote{Fica como dever de casa pra você.}
\\
\\
Existem equivalências que são tão importantes que possuem um nome particular. As duas abaixo são chamadas de Lei de DeMorgan.

\begin{center}
$\sim(P \land Q) = (\sim P) \lor (\sim Q)$ \\
$\sim(P \lor Q) = (\sim P) \land (\sim Q)$
\end{center}

A tabela-verdade da primeira parte é:
\begin{center}
\begin{tabular}{ c c || c c c || c c }
P & Q & $\sim P$ & $\sim Q$ & $P \land Q$ & $\sim (P \land Q)$ & $(\sim P) \lor (\sim Q)$ \\
\hline
V & V & F & F & V & F & F \\
V & F & F & V & F & V & V \\
F & V & V & F & F & V & V \\
F & F & V & V & F & V & V
\end{tabular}
\end{center}

A tabela-verdade da segunda parte é:
\begin{center}
\begin{tabular}{ c c || c c c || c c }
P & Q & $\sim P$ & $\sim Q$ & $P \lor Q$ & $\sim (P \lor Q)$ & $(\sim P) \land (\sim Q)$ \\
\hline
V & V & F & F & V & F & F \\
V & F & F & V & V & F & F \\
F & V & V & F & V & F & F \\
F & F & V & V & F & V & V
\end{tabular}
\end{center}

Nessa parte o professor elenca várias equivalências que podem ser verificadas por suas respectivas tabelas-verdade.

\begin{itemize}
\item $ P \implies Q = (\sim Q) \implies (\sim P) $ - Lei Contrapositiva
\item $ \sim (P \land Q) = \  (\sim P) \ \lor \ (\sim Q) $ - Lei de DeMorgan 1
\item $ \sim (P \lor Q) = \  (\sim P) \ \land \ (\sim Q) $ - Lei de DeMorgan 2
\item $ P \land Q = Q \land P $ - Lei Comutativa 1
\item $ P \lor Q = Q \lor P $ - Lei Comutativa 2
\item $ P \land (Q \lor R) = (P \land Q) \lor (P \land R)$ - Lei Distributiva 1
\item $ P \lor (Q \land R) = (P \lor Q) \land (P \lor R) $ - Lei Distributiva 2
\item $ P \land (Q \land R) = (P \land Q) \land R $ - Lei Associativa 1
\item $ P \lor (Q \lor R) = (P \lor Q) \lor R $ - Lei Associativa 2
\end{itemize}

\textbf{Comentário}: Como são muitas equivalências, se eu colocar todas as tabelas-verdade aqui vai ficar uma poluição muito grande. Então cabe a você demonstrar todas essas equivalências por conta própria. Não confie só porque está escrito ai. Demonstre com seu próprio esforço todas elas.

\section{Quantificadores}

Até agora nós já vimos um punhado de símbolos que nos permitem construir proposições consideravelmente complexas, bem como, achar as equivalências entre diferentes proposições. Mas ainda temos um problema para superar: o que fazer quando temos que lidar com infinitos elementos?
\\
\\
O professor usa o seguinte exemplo: Imagine um conjunto infinito de números inteiros $X = \{ x_1,x_2,x_3, \dots \}$. Com os símbolos apresentados até agora, como faríamos para representar as seguintes proposições?

\begin{itemize}
\item $Z_1:$ "Todos os elementos de $X$ são ímpares"
\item $Z_2:$ "Pelo menos um elemento de $X$ é ímpar"
\end{itemize}

Primeiro vamos definir a proposição $P :$ "O número $x$ é ímpar". Para a primeira proposição ($Z_1$), teríamos algo parecido com 

$$P(x_1) \land P(x_2) \land P(x_3) \land \dots$$

E para a segunda proposição ($Z_2$), algo como

$$P(x_1) \lor P(x_2) \lor P(x_3) \lor \dots$$

Ou seja, acabaríamos com proposições de tamanho infinito.
\\
\\
Para superar essa dificuldade, vamos introduzir novos símbolos "$\forall$" \ e "$\exists$". O símbolo $\forall$ significa "para todos" \ e o símbolo $\exists$ significa "existe algum". Desse modo podemos reescrever as proposições anteriores como:

\begin{itemize}
\item $Z_1: \forall \ x \in X, P(x)$
\item $Z_2: \exists \ x \in X, P(x)$
\end{itemize}

Nós chamamos esses novos símbolos de \textbf{quantificadores}.\footnote{Não sei você, mas pra mim esse daria um bom nome de banda.} Esse nome é porque eles se referem a alguma propriedade da quantidade de algo. $\forall$ é chamado de \textbf{quantificador universal} e $\exists$ é chamado de \textbf{quantificador existencial}. Proposições que contenham esses símbolos são chamadas de \textbf{proposições quantificadas}.
\\
\\
\textbf{Dica}: Leia o exemplo 2.5 do livro. O professor Hammack dá 4 exemplos em inglês e "traduz" \ esses exemplos para os símbolos que já aprendemos até agora.

\section{Um pouco mais de Proposições Condicionais}

Na seção 2.1 nós vimos que uma \textbf{sentença aberta} é uma proposição cuja validade depende do valor da variável. Contudo, uma característica que os quantificadores possuem é transformar sentenças abertas em proposições.
\\
\\
Dada a sentença aberta "$x$ é par $\implies$ $x$ é múltiplo de 6". A validade dessa afirmação depende totalmente do valor dado para a variável $x$. Agora, quando adicionamos um quantificador "$\forall \ x \in \mathbb{Z}, x$ é par $\implies x$ é múltiplo de 6", ela se torna uma proposição, ou seja, podemos definir se ela é verdadeira ou falsa (nesse caso, falsa).
\\
\\
De maneira geral, dadas duas proposições $Q(x)$ e $P(x)$, a expressão $\forall \ x \in \mathbb{Z}, P(x) \implies Q(x)$ é verdadeira ou falsa. Logo, ela é uma proposição e não uma sentença aberta.
\\
\\
Agora chegamos no ponto importante sobre esse assunto: sempre que você ver duas proposições a respeito de uma variável que seja elemento de algum conjunto (previamente definido), uma expressão da forma $P(x) \implies Q(x)$ deve ser interpretada como sendo uma proposição do tipo $\forall \ x \in Conjunto, P(x) \implies Q(x)$. Ou seja, se uma proposição condicional não estiver quantificada explicitamente, então existe um quantificador implícito atrelado a ela. O motivo desse quantificador não aparecer é que cansa ter que ficar escrevendo toda hora "$\forall \ x \in Conjunto$".
\\
\\
Agora nós vamos definir de maneira mais geral (mas consistente com o que foi dito na seção 2.1) as proposições condicionais.

\begin{definition}[Proposição Condicional]
Se $P$ e $Q$ são proposições ou sentenças abertas, então "Se $P$, então $Q$" \  será uma proposição.\\
Essa proposição será verdadeira se for impossível que $P$ seja verdadeira enquanto $Q$ seja falsa. E será falsa se existir alguma situação onde $P$ é verdadeira e $Q$ é falsa.
\end{definition}

\section{Traduzindo Português para Lógica Simbólica}

Quando estamos lendo provas de teoremas, sempre devemos estar atentos para a estrutura lógica e o significado das sentenças. Não é raro ter que converter as expressões feitas de palavras para expressões feitas dos símbolos que estudamos até agora. Essa seção é uma prática para você exercitar essa competência.
\\
\\
\textbf{Comentário}: A partir desse ponto pode ser que você veja alguns conceitos que não foram previamente explicados no material. Não se assuste por causa disso. Foque apenas na lógica e no que você aprendeu até aqui.
\\
\\
\textbf{Exemplo 1}: O teorema do valor médio do Cálculo Infinitesimal.
\\
\hrule
\vspace{5pt}
Se $f$ é contínua no intervalo $[a,b]$ e diferenciável em $(a,b)$, então existe um número $c \in (a,b)$ onde $f'(c) = \dfrac{f(b) - f(a)}{b - a}$.
\vspace{5pt}
\hrule
\vspace{5pt}
A tradução para a forma simbólica dessa afirmação é:
$$ \left( (f \ cont. \ em \  [a,b]) \land (f \ dif. \ em \  (a,b)) \right) \implies \left(\exists \ c \in (a,b), f'(c) = \dfrac{f(b) - f(a)}{b - a} \right) $$

Eu sei, ficou meio poluído. Pra melhorar um pouco vamos definir novas proposições e limpar um pouco essa sujeira toda. $P : $"$f$ é contínua no intervalo $[a,b]$", $Q : $"$f$ é diferenciável em $(a,b)$".

$$ \left( P \land Q \right) \implies \left(\exists \ c \in (a,b), f'(c) = \dfrac{f(b) - f(a)}{b - a} \right) $$
\\
\\
\textbf{Exemplo 2}: A conjectura de Goldbach.
\\
\hrule
\vspace{5pt}
Todo número inteiro par maior que $2$ é a soma de dois primos.
\vspace{5pt}
\hrule
\vspace{5pt}
A tradução para a forma simbólica dessa afirmação é:
$$ \left( n \in X \right) \implies \left( \exists \ p,q \in P, n = p + q \right) $$
Ou então, podemos usar essa outra forma equivalente:
$$ \forall \ n \in X, \exists \ p,q \in P, n = p + q $$
\\
Sendo $P = \{x : x \ é \ primo \}$ e $X = \{ 4,6,8,10.\dots \}$. 
\\
\\
Esse exemplo acima nos mostra uma relação interessante entre as proposições do tipo $(n \in X) \implies Q(n)$ e as do tipo $\forall \ n \in X, Q(x)$. Toda proposição universalmente quantificada\footnote{Ou seja, com o uso do $\forall$.} pode ser expressa como uma proposição condicional.


\begin{fact}[Equivalência de proposições]
Suponha que $X$ é um conjunto e $Q(x)$ é uma proposição a respeito de cada $x \in X$. As proposições abaixo são equivalentes:
$$ \forall \ x \in X, Q(x) $$
$$ (x \in X) \implies Q(x) $$
\end{fact}

Essas equivalências são importantes porque podemos trocar a maneira como um teorema se apresenta na hora de tentar provar ou desprovar uma proposição.
\\
\\
\textbf{Aviso}: Na hora que você ler um teorema, evite ir direto para as palavras "ou", "se", "e" \ como se elas sempre significassem os seus respectivos valores lógicos. O contexto é quem diz o real significado das palavras. Sempre analise o enunciado inteiro antes de transcrever para a linguagem simbólica da lógica.

\section{Negando Proposições}

Não é raro que ao tentar demonstrar uma proposição complexa $R$, por exemplo, seja mais fácil trabalhar com a negação dessa proposição $\sim R$. Mas na frente do curso você verá que essa é uma das técnicas usadas para provar teoremas.
\\
\\
Também não é incomum termos que lidar com negações de proposições quantificadas. Considere a proposição $\sim(\forall \ x \in \mathbb{N}, P(x))$. Podemos ler como "Não é o caso que $P(x)$ é verdade para todos os $x$ números naturais". Ou seja, a negativa dessa proposição envolve outro quantificador (o de existência). De modo que podemos reescrever essa negação como $\exists \ x \in \mathbb{N},\sim P(x)$.
\\
\\
Chegamos em um ponto importante aqui. Os quantificadores são usados para negar um ao outro. Observe as seguintes equivalências:

$$ \sim (\forall \ x \in X, P(x)) = \exists \ x \in X, \sim P(x) $$
$$ \sim (\exists \ x \in X, P(x)) = \forall \ x \in X, \sim P(x) $$

\textbf{Comentário}: Se você leu com atenção até aqui, você deve ser capaz de compreender essas equivalências. Se elas não fazem sentido na sua cabeça, me manda uma dm no \href{https://twitter.com/bruno_ruas2}{twitter} explicando qual a sua dúvida. Eu posso melhorar o texto com o seu feedback.
\\
\\
Quando se está escrevendo demonstrações, as vezes é necessário negar proposições condicionais $P \implies Q$. Se você aprendeu a tabela-verdade desse tipo de operador, você sabe que a única maneira de ele ser falso é quando temos $P$ verdadeiro e $Q$ falso. Em termo da lógica dizemos que $\sim (P \implies Q) = P \land \sim Q$. 
\\
\\
\textbf{Dica}: Tente fazer o exercício 5 da seção 2.10. Tem a solução dele no final do livro. Se mesmo assim você não conseguir entender, comenta sua dúvida na postagem dessa aula no \href{https://economiamainstream.com.br/artigo/matematica/}{Economia Mainstream}.

\section{Inferência Lógica}

A inferência lógica é o processo de se tirar conclusões dadas premissas e regras previamente estabelecidas. Existem algumas maneiras de se trabalhar esse processo de pensamento. O professor elenca 3 modos: modus ponens, modus tollens e eliminação.

\begin{align*}
Modus \ Ponens && Modus \ Tollens && Elimina\c{c}\~{a}o \\
P \implies Q & & P \implies Q     & & P \lor Q \\
P            & & \sim Q           & & \sim P   \\
\dfrac{\hspace*{50pt}}{Q}  & & \dfrac{\hspace*{50pt}}{\sim P} & & \dfrac{\hspace*{50pt}}{Q}
\end{align*}


Os diagramas acima são fáceis de se ler. Tudo que está em cima da linha é dado como verdadeiro. O que está abaixo é a consequência lógica do que está acima. É importante que você internalize esses métodos de pensamento. Você usará bastante ao longo da sua jornada pela matemática.

\section{Nota Importante}

Para finalizar, o professos elenca 3 grandes motivos para se estudar lógica: 1) As tabelas-verdade nos dizem exatamente os significados das palavras "e", "ou", "não", "se...então", etc; 2) As regras da inferência produzem um sistema onde podemos deduzir novas informações de afirmações anteriores; 3) Regras lógicas (como a lei de DeMorgan) nos ajudam a transformar proposições em formatos que sejam mais fáceis de se trabalhar e que são equivalentes.
\\
\\
Lógica é a linguagem-comum que toda a Matemática utiliza, então temos que ter uma boa base para poder escrever e entender a própria matemática. Durante o resto desse livro não usaremos com tanta frequência os símbolos que vimos $(\land, \lor, \implies, \iff, \sim, \forall, \exists)$ mas sempre que ler uma passagem matemática, você deve usa-los, mentalmente ou em papel, para se certificar que você a compreendeu de uma maneira inequívoca e verdadeira.

%%%%%%%%%%%%%%%%%%%%%%%%%%%%%%%%%%%%%%%%%%%%%%%%%%%
%                 CHAPTER                         %
%%%%%%%%%%%%%%%%%%%%%%%%%%%%%%%%%%%%%%%%%%%%%%%%%%%
\chapter{Contagem}


\begin{chapquote}{página 65}
	``Counting can become quite subtle, and in this chapter we explore some of its more sophisticated aspects. Our goal is still to answer the question 'How many?' but we introduce mathematical techniques that bypass the actual process of counting individual objects''
\end{chapquote}

\section{Listas}

Uma \textbf{lista} é uma sequência ordenada de objetos. Esses objetos são mantidos entre um par de parênteses e separados por vírgulas. Os objetos dentro de uma lista são chamados de \textbf{entradas}\footnote{Do original, \textbf{entries}}. Já que uma lista é uma sequência ordenada, é evidente que a ordem dos seus elementos é suficiente para distinguir listas que contenham os mesmo objetos.

$$ (a,b,c) \neq (c,b,a) $$

\textbf{Comentário}: Também é comum escrever uma lista sem os parênteses e as vírgulas. Esse formato de escrita se chama \textbf{string}. Nesse caso $(a,b,c)$ é a mesma coisa de $abc$. Essa outra maneira só é usada quando não há risco de confusão entre as entradas da lista. Fica ligado e mantém essas duas formas de escrita como padrão.
\\
\\
Como já vimos conjuntos no capítulo 01, podemos usá-los para comparação. No caso dos conjuntos a ordem dos elementos não importa na comparação. Ou seja, $\{a,b,c\} = \{c,b,a\}$. Já vimos acima que essa propriedade não é mantida nas listas. 
\\
\\
Diferentemente dos conjuntos, uma lista pode ter entradas repetidas sem nenhum problema. A lista $(a,a,a,a,b)$, por exemplo, é perfeitamente aceitável.
\\
\\
Tal qual a cardinalidade dos conjuntos, nós contamos quantas entradas existem em uma lista. Chamamos essa medida de \textbf{comprimento}. A lista $(a,a,a,a,b)$ possui um comprimento de 5.
\\
\\
\textbf{Regra}: Duas listas são \textbf{iguais} se possuírem exatamente as mesmas entradas nas mesmas posições. Ou seja, também possuem o mesmo comprimento.
\\
\\
Só existe uma lista cujo comprimento é igual a zero. Denominamos essa lista de \textbf{lista vazia}. Denotada por $()$.\footnote{Sim, lembra muito o conceito de conjunto vazio.}

\section{O Princípio da Multiplicação}

Existem muitos problemas práticos que envolvem a contagem do número possível de listas que satisfaz uma determinada condição ou propriedade.

\begin{fact}[Princípio da Multiplicação]
Suponha que em uma lista de comprimento $n$ exista $a_1$ escolhas possíveis para a primeira entrada, $a_2$ escolhas possíveis para a segunda entrada, $a_3$ escolhas possíveis para a terceira entrada e etc. Então, o total de listas diferentes que podem ser geradas por essas entradas será igual ao produto entre $a_1 \times a_2 \times a_3 \times \dots \times a_n$. 
\end{fact}

\textbf{Dica}: Nas páginas 67 e 68 do livro, o professor coloca dois exemplos que tornarão esse conceito abaixo bem mais entendível. Se você não entender como esse fato é evidente, dá uma olhada no livro e volta aqui.
\\
\\
Embora no livro não seja dada uma demonstração desse princípio. Eu acho que podemos tentar provar que essa afirmação é verdadeira. Não se preocupe se você não conseguir entender essa demonstração agora. Volte quando estiver mais adiantado no curso e tente novamente.

\begin{demonstration}[Princípio da Multiplicação\footnote{JOHNSTON, William; MCALLISTER, Alex. A transition to advanced mathematics: a survey course. OUP USA, 2009. p. 365.\\\href{https://math.stackexchange.com/questions/3053969/using-induction-to-prove-the-multiplication-rule}{Link de onde veio a ideia para a parte para $P(m)$ com $m > 2$.}}]

Começaremos com a proposição $P(m)$: "Se existirem $A_m$ conjuntos com $n_m$ elementos em cada conjunto. O Cardinal do produto cartesiano de todos os $m$ conjuntos será a multiplicação de todos os $m$ cardinais $|A_m|$". 
\\
\\
Essa proposição é equivalente ao enunciado do princípio da multiplicação.\footnote{Você consegue ver essa equivalência?}
\\
\\
É trivial ver que uma lista de comprimento $n = 1$ tem exatamente $a_1$ possíveis listas geradas por todas as $a_1$ possibilidades de entradas nessa lista. Então não precisamos nos preocupar com a proposição $P(1)$.
\\
\\
$P(2)$: "Se $A_1$ e $A_2$ são conjuntos finitos onde $A_1$ contém $n_1$ elementos e $A_2$ contém $n_2$ elementos, então, o conjunto $A_1 \times A_2$ possui $n_1 \cdot n_2$ elementos".\footnote{Ou seja, o princípio da multiplicação nada mais é que uma proposição a respeito do cardinal do produto cartesiano para conjuntos finitos.}
\\
\\
A demonstração dessa proposição é simples. Para cada um dos $n_1$ elementos $a \in A_1$ na primeira coordenada do par ordenado $(a,b) \in A_1 \times A_2$, existem exatamente $n_2$ elementos $b \in A_2$. Uma vez que existem $n_1$ elementos em $A_1$ que podem ser essa primeira coordenada do par ordenado $(a,b)$, então, existem ao todo $n_1 \cdot n_2$ possíveis pares ordenados no produto cartesiano $A_1 \times A_2$.
\\
\\
Agora vamos fazer uma pequena adaptação nessa demonstração para qualquer quantidade de conjuntos, ou seja, para qualquer $P(m)$ cujo $m > 2$.
\\
\\
Suponha que agora temos 3 conjuntos: $A_1$, $A_2$ e $A_3$. Para podermos demonstrar $P(3)$: "Se $A_1 \times A_2 \times A_3$, então o cardinal será  $n_1 \cdot n_2 \cdot n_3$"\  só precisamos da seguinte linha de pensamento: Podemos definir um novo conjunto $B = A_1 \times A_2$. Desse modo, podemos reescrever o cardinal anterior como $B \times A_3$. Essa nova reescrita possui apenas dois elementos. Portanto, podemos usar a proposição já demonstrada $P(2)$ sem nenhum prejuízo.
\\
\\
Aplicando esse mesmo procedimento para qualquer $n > 2$ fica demonstrado que o cardinal de quaisquer conjuntos $A_n$ para qualquer $n \in \mathbb{N}$ será a multiplicação do cardinal dos conjuntos $A_n$. Aplicando a equivalência da proposição $P(n)$ com o princípio da multiplicação, finalizamos a demonstração. $\blacksquare$
\end{demonstration}

Existem dois tipos de problemas que envolvem contagem de listas: Problemas que possuem entradas repetidas e Problemas que não permitem entradas repetidas. Nós podemos chamar as listas do segundo tipo de problema de \textbf{listas não repetitivas}.
\\
\\
Usando o princípio da multiplicação você é capaz de resolver todos os problemas envolvendo contagens de listas sem precisar ficar escrevendo as soluções possíveis, ao invés disso, você só precisa interpretar as opções de entradas na lista e usar a multiplicação.
\\
\\
\textbf{Dica}: Tenta fazer os exemplos 3.1, 3.2 e 3.3 da página 69 até a 72.

\section{Os Princípios da Adição e Subtração}

Vamos ver mais dois princípios de contagem. Você já está familiarizado com eles mas agora definiremos esses princípios usando a linguagem dos conjuntos.

\begin{fact}[Princípio da Adição]
Suponha que um conjunto finito $X$ pode ser decomposto na união $X = \bigcup_{i = 1}^{n}\limits X_i$ onde $ X_i \cap X_j = \emptyset \ \forall \ i \neq j$. Então, $|X| = \sum_{i = 1}^{n}\limits |X_i|$.
\end{fact}

Calma. Não é difícil de entender. O que estamos dizendo ai é: "Se $X$ é um conjunto formado por $n$ outros conjuntos menores, e nenhum desses conjuntos possui interseção entre si. Então, o cardinal de $X$ será a soma dos cardinais de todos os $n$ subconjuntos. A única novidade nessa notação é o sigma para denominar somatório.
\\
\\
\textbf{Dica}: Veja os exemplos 3.5 e 3.6 na página 75 para ter uma ideia da aplicação desses conceitos na prática.

Agora vamos ver o princípio da subtração. Você não deve ter grandes dificuldades de entender esse conceito.

\begin{fact}[Princípio da Subtração]
Se $X$ é um subconjunto de um conjunto finito $U$, então $|\overline{X}| = |U| - |X|$. Ou seja, se $X \subseteq U$, então $|U - X| = |U| - |X|$.
\end{fact}

Existem situações onde é mais fácil contar o total de um conjunto maior, e retirar uma parte desse total que não queremos, do que contar diretamente a parte desejada. Eu sei, tá um pouco confuso. Mas a ideia é simples: Usamos esse método para computar o que sobra após a retirada de algumas opções.

\textbf{Dica}: Lê o exemplo 3.7 novamente. A gente usa exatamente essa abordagem pra chegar no resultado.

\section{Fatoriais e Permutações}
\section{Contando Subconjuntos}
Selecionando $k$ elementos de um conjunto $A$ com $|A| = n$, quantos subconjuntos $A_i$ serão formados dessa permutação?
\\
Lembre-se:
\\
$$\frac{n!}{(n-k)!} > i$$
\\
Isso ocorre pois em listas que contenham os mesmo elementos a ordem que esses elementos são dispostos dentro de cada uma faz com que sejam listas diferentes,então $(a,b)\neq(b,a)$. O que é diferente nos conjuntos, pois dados dois conjuntos $A = \lbrace a,b \rbrace$ e $B = \lbrace b,a \rbrace$, $A=B$.
\\
\begin{definition}
	Se $k,n \in \mathbb{Z}$, então $n \choose k$ denota o número de subconjuntos que podem ser feitos escolhendo $k$ elementos de um conjunto com $n$ elementos. Lemos $n \choose k$ como o "número de permutações de $n$ termos $k$ a $k$"
\end{definition}
Suponha um conjunto $D = \bigcup_{i = 1}^{j}\limits D_i$, onde $|D| = n$. Agora permute cada subconjuntos $D_i$ em listas de $k$ elementos, montando uma tabela temos:
\begin{center}
\begin{tabular}{ c || c || c || c || c }
$D_1$ & $D_2$ & $D_3$ & $\cdots$ & $D_j$ \\
\hline
$Lista_1^{D_1}$ & $Lista_1^{D_2}$ & $Lista_1^{D_3}$ & $\cdots$ & $Lista_1^{D_j}$\\
$Lista_2^{D_1}$ & $Lista_2^{D_2}$ & $Lista_2^{D_3}$ & $\cdots$ & $Lista_2^{D_j}$\\
$Lista_3^{D_1}$ & $Lista_3^{D_2}$ & $Lista_3^{D_3}$ & $\cdots$ & $Lista_3^{D_j}$\\
$\vdots$ & $\vdots$ & $\vdots$ & $\vdots$ & $\vdots$\\
$Lista_{k!}^{D_1}$ & $Lista_{k!}^{D_2}$ & $Lista_{k!}^{D_3}$ & $\cdots$ & $Lista_{k!}^{D_j}$\\
\end{tabular}
\end{center}
A tabela acima possui dimensões de $k! \times$ $ n \choose k$. Perceba que pelo \textbf{princípio da soma}:
\\
$$P(n,k) = \sum\limits_{i=1}^j P(|D_i|, k) = k! \times {n \choose k}$$
\\
Portanto:
\\
$$\frac{n!}{(n-k)!} = k! \times {n \choose k}$$
\\
Dividimos ambos os lados por $k!$ para chegarmos à uma fórmula geral para calcularmos o número de subconjuntos com $k$ elementos de um conjunto qualquer com $n$ elementos.
$$\frac{n!}{k!(n-k)!} = {n \choose k}$$
\begin{fact}[Contagem de subconjuntos]
Se $0 \leqslant k \leqslant n$, então ${n \choose k} = \frac{n!}{k!(n-k)!}$. Caso contrário ${n \choose k} = 0$
\end{fact}
\section{O Triângulo de Pascal e O Teorema Binomial}
\section{O Princípio da Inclusão-Exclusão}
\section{Contando Multiconjuntos}
\section{Os Princípios da Divisão e da Casa dos Pombos}
\section{Prova Combinatorial}

%%%%%%%%%%%%%%%%%%%%%%%%%%%%%%%%%%%%%%%%%%%%%%%%%%%
%                    PART                         %
%%%%%%%%%%%%%%%%%%%%%%%%%%%%%%%%%%%%%%%%%%%%%%%%%%%
\part{Como Provar Afirmações Condicionais}

%%%%%%%%%%%%%%%%%%%%%%%%%%%%%%%%%%%%%%%%%%%%%%%%%%%
%                 CHAPTER                         %
%%%%%%%%%%%%%%%%%%%%%%%%%%%%%%%%%%%%%%%%%%%%%%%%%%%
\chapter{Prova Direta}

%%%%%%%%%%%%%%%%%%%%%%%%%%%%%%%%%%%%%%%%%%%%%%%%%%%
%                 CHAPTER                         %
%%%%%%%%%%%%%%%%%%%%%%%%%%%%%%%%%%%%%%%%%%%%%%%%%%%
\chapter{Prova Contra-positiva}

%%%%%%%%%%%%%%%%%%%%%%%%%%%%%%%%%%%%%%%%%%%%%%%%%%%
%                 CHAPTER                         %
%%%%%%%%%%%%%%%%%%%%%%%%%%%%%%%%%%%%%%%%%%%%%%%%%%%
\chapter{Prova por Contradição}


%%%%%%%%%%%%%%%%%%%%%%%%%%%%%%%%%%%%%%%%%%%%%%%%%%%
%                    PART                         %
%%%%%%%%%%%%%%%%%%%%%%%%%%%%%%%%%%%%%%%%%%%%%%%%%%%
\part{Mais Sobre Provas}

%%%%%%%%%%%%%%%%%%%%%%%%%%%%%%%%%%%%%%%%%%%%%%%%%%%
%                 CHAPTER                         %
%%%%%%%%%%%%%%%%%%%%%%%%%%%%%%%%%%%%%%%%%%%%%%%%%%%
\chapter{Prova de Afirmações Não-Condicionais}

%%%%%%%%%%%%%%%%%%%%%%%%%%%%%%%%%%%%%%%%%%%%%%%%%%%
%                 CHAPTER                         %
%%%%%%%%%%%%%%%%%%%%%%%%%%%%%%%%%%%%%%%%%%%%%%%%%%%
\chapter{Provas Envolvendo Conjuntos}

%%%%%%%%%%%%%%%%%%%%%%%%%%%%%%%%%%%%%%%%%%%%%%%%%%%
%                 CHAPTER                         %
%%%%%%%%%%%%%%%%%%%%%%%%%%%%%%%%%%%%%%%%%%%%%%%%%%%
\chapter{Contraprova}

%%%%%%%%%%%%%%%%%%%%%%%%%%%%%%%%%%%%%%%%%%%%%%%%%%%
%                 CHAPTER                         %
%%%%%%%%%%%%%%%%%%%%%%%%%%%%%%%%%%%%%%%%%%%%%%%%%%%
\chapter{Indução Matemática}


%%%%%%%%%%%%%%%%%%%%%%%%%%%%%%%%%%%%%%%%%%%%%%%%%%%
%                    PART                         %
%%%%%%%%%%%%%%%%%%%%%%%%%%%%%%%%%%%%%%%%%%%%%%%%%%%
\part{Relações, Funções e Cardinalidade}

%%%%%%%%%%%%%%%%%%%%%%%%%%%%%%%%%%%%%%%%%%%%%%%%%%%
%                 CHAPTER                         %
%%%%%%%%%%%%%%%%%%%%%%%%%%%%%%%%%%%%%%%%%%%%%%%%%%%
\chapter{Relações}

%%%%%%%%%%%%%%%%%%%%%%%%%%%%%%%%%%%%%%%%%%%%%%%%%%%
%                 CHAPTER                         %
%%%%%%%%%%%%%%%%%%%%%%%%%%%%%%%%%%%%%%%%%%%%%%%%%%%
\chapter{Funções}

%%%%%%%%%%%%%%%%%%%%%%%%%%%%%%%%%%%%%%%%%%%%%%%%%%%
%                 CHAPTER                         %
%%%%%%%%%%%%%%%%%%%%%%%%%%%%%%%%%%%%%%%%%%%%%%%%%%%
\chapter{Provas com Cálculo Infinitesimal}

%%%%%%%%%%%%%%%%%%%%%%%%%%%%%%%%%%%%%%%%%%%%%%%%%%%
%                 CHAPTER                         %
%%%%%%%%%%%%%%%%%%%%%%%%%%%%%%%%%%%%%%%%%%%%%%%%%%%
\chapter{Cardinalidade de Conjuntos}

\end{document}
