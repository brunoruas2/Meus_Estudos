%%%%%%%%%%%%%%%%%%%%%%%%%%%%%%%%%%%%%%%%%%%%%%%%%%%
%% LaTeX book template                           %%
%% Author:  Amber Jain (http://amberj.devio.us/) %%
%% License: ISC license                          %%
%%%%%%%%%%%%%%%%%%%%%%%%%%%%%%%%%%%%%%%%%%%%%%%%%%%

\documentclass[a4paper,11pt,oneside]{book}

% pacotes utilizados
\usepackage[T1]{fontenc}
\usepackage[utf8]{inputenc}
\usepackage{lmodern}
\usepackage{hyperref}
\usepackage{graphicx}
\usepackage[portuguese]{babel}
\usepackage{amsfonts}
\usepackage[usenames,dvipsnames]{xcolor}
\usepackage{mathtools}
\usepackage{amssymb} % alguns simbolos matematicos
\usepackage{mathrsfs} % letras cursivas
\usepackage{listings} % coding examples in latex file
\usepackage{xcolor} % changing colors
\usepackage{amsthm} % definir estilo do teorema
\usepackage[hang,flushmargin]{footmisc} % remove footnote's identation
\usepackage{cancel} % para poder colocar o tracinho de cancelamento
\usepackage{tikz} % to draw Venn's diagrams
\usepackage{amsmath} % to write a function by cases

% dark theme no pdf
\pagecolor[rgb]{0.1,0.1,0.1} %black
\color[rgb]{0.9,0.9,0.9} %grey

% criando o modelo de definicoes/teoremas/fatos/demonstracoes
\theoremstyle{definition}

\newtheoremstyle{break}% name
  {10pt}%         Space above, empty = `usual value'
  {10pt}%         Space below
  {}% Body font
  {}%         Indent amount (empty = no indent, \parindent = para indent)
  {\bfseries}% Thm head font
  {}%        Punctuation after thm head
  {\newline}% Space after thm head: \newline = linebreak
  {}%         Thm head spec

\theoremstyle{break}

% definindo as categorias de formalidade
\newtheorem{definition}{Definição}[section]
\newtheorem{fact}{Fato}[section]
\newtheorem{demonstration}{Demonstração}[section]
\newtheorem{theorem}{Teorema}

% tirando identação dos paragrafos
\setlength{\parindent}{0ex}

% setting das cores quando usar codigo python
\definecolor{codegreen}{rgb}{0,0.6,0}
\definecolor{codegray}{rgb}{0.5,0.5,0.5}
\definecolor{codepurple}{rgb}{0.58,0,0.82}
\definecolor{backcolour}{rgb}{0.95,0.95,0.92}

\lstdefinestyle{mystyle}{
    backgroundcolor=\color{backcolour},   
    commentstyle=\color{codegreen},
    keywordstyle=\color{magenta},
    numberstyle=\tiny\color{codegray},
    stringstyle=\color{codepurple},
    basicstyle=\ttfamily\footnotesize,
    breakatwhitespace=false,         
    breaklines=true,                 
    captionpos=b,                    
    keepspaces=true,                 
    numbers=left,                    
    numbersep=5pt,                  
    showspaces=false,                
    showstringspaces=false,
    showtabs=false,                  
    tabsize=2
}

\lstset{style=mystyle}

% dedicatoria 
% Source: http://www.tug.org/pipermail/texhax/2010-June/
\newenvironment{dedication}
{
   \cleardoublepage
   \thispagestyle{empty}
   \vspace*{\stretch{1}}
   \hfill\begin{minipage}[t]{0.66\textwidth}
   \raggedright
}
{
   \end{minipage}
   \vspace*{\stretch{3}}
   \clearpage
}

% Chapter quote at the start of chapter        %
% Source: http://tex.stackexchange.com/a/53380 %
\makeatletter
\renewcommand{\@chapapp}{}% Not necessary...
\newenvironment{chapquote}[2][2em]
  {\setlength{\@tempdima}{#1}%
   \def\chapquote@author{#2}%
   \parshape 1 \@tempdima \dimexpr\textwidth-2\@tempdima\relax%
   \itshape}
  {\par\normalfont\hfill--\ \chapquote@author\hspace*{\@tempdima}\par\bigskip}
\makeatother


%%%%%%%%%%%%%%%%%%%%%%%%%%%%%%%%%%%%%%%%%%%%%%%%%%%
% First page of book which contains 'stuff' like: %
%  - Book title, subtitle                         %
%  - Book author name                             %
%%%%%%%%%%%%%%%%%%%%%%%%%%%%%%%%%%%%%%%%%%%%%%%%%%%
% Book's title and subtitle
\title{\Huge \textbf{Microeconomia} \\ 
\Large Tradução da 9 edição \\
\huge Hal R. Varian}

% Author
\author{
\textsc{Resumo e Adaptação por:} \\
\textsc{Bruno de M. Ruas}
}

\begin{document}

\frontmatter
\maketitle

\tableofcontents
%\listoffigures
%\listoftables

\mainmatter

%%%%%%%%%%%%%%%%%%%%%%%% PART %%%%%%%%%%%%%%%%%%%%%%%%
\part{Preparativos}

%%%%%%%%%%%%%%%%%%%%%%%% CHAPTER %%%%%%%%%%%%%%%%%%%%%%%%
\chapter{Matemática}

\begin{chapquote}{página 1.008}
	``Revisão breve de alguns conceitos matemáticos utilizados no texto''.
\end{chapquote}

Bem vindo ao meu resumo do livro do prof. Varian. Ao contrário do que ele fez, eu preferi trazer o apêndice de matemática pro começo do material porque aqui nós vamos ver as ferramentas que serão usadas para a explicação dos conceitos teóricos ao longo do material.
\\
\\
Aqui a gente só vai dar um overview básico nos conceitos. Não tenha dúvida que alguém mais experimentado em matemática torceria o nariz pra algumas definições dadas aqui. Mas o objetivo é te dar um "norte"\ a respeito de alguns conceitos normalmente usados. Não se assuste com a simplicidade de algumas coisas. Melhor garantir agora do que sofrer mais pra frente no texto.

\section{Funções}

Sejam dois números quaisquer $x$ e $y$, uma \textbf{função} ou \textbf{transformação} é uma regra que descreve uma relação entre eles.
\\
\\
Para demonstrar que existe alguma dependência entre duas variáveis usamos a notação $y = f(x)$, onde nossa variável $y$ (chamada de \textbf{dependente}) é o resultado de alguma transformação (denotada pelo símbolo $"f"$) realizada em $x$ (nossa variável \textbf{independente}).
\\
\\
Não é raro ter uma variável dependente relacionada a várias outras variáveis. Nesses casos é comum o uso da notação anterior com a adição das novas incógnitas. Algo como $y = f(x_1,x_2,...,x_n)$.

\section{Gráficos}

Não tem muito o que falar aqui. Dá uma lida lá na página 1010.

\section{Propriedades de funções}

Uma função pode ter algumas características que facilitam a sua descrição. Aqui temos algumas que serão usadas ao longo do curso:
\\
\\
Uma \textbf{função contínua} é aquela que não possui nenhum "salto"\ ou "quebra". 
\\
\\
Uma \textbf{função suave} é aquela que não tem "dobras"\ nem "cantos".
\\
\\
Uma \textbf{função monotônica} é aquela que sempre segue o mesmo sentido (ou crescendo ou decrescendo) sem nunca mudar de sentido. 
Quando é crescente a medida que $x$ cresce, chamaremos de \textbf{função monotônica crescente}. Quando descrescer a medida que $x$ crescer, chamaremos de \textbf{função monotônica decrescente}.

\section{Funções inversas}

Uma das implicações de quando uma função é monotônica é que, para cada $x$, sempre existirá apenas um único $y$ associado. 
\\
\\
Uma \textbf{função inversa} é a função que, sempre que colocarmos um $y$ como variável independente teremos como resultado um $x$ de alguma função anterior.\footnote{Eu tentei não deixar confuso mas se ficou com dúvida, pesquisa um pouco sobre o tema.}

\section{Equações e identidades}

Podemos relacionar dois ou mais elementos por meio do uso de \textbf{equações} (usando o símbolo da igualdade "$=$"). Onde as suas respectivas \textbf{soluções} são os valores atribuíveis as incógnitas que assegurem a validade da relação proposta.
\\
\\
Uma \textbf{identidade} (que tem o símbolo dado por "$\equiv$") é um tipo de relação onde sempre haverá as soluções independentemente de quais valores suas variáveis assumam.

\section{Funções lineares}

Chamamos de \textbf{função linear}, qualquer função da forma $y = ax + b$. Fique atento porque uma função linear pode ser expressa de maneira implícita (ou seja, será necessário desenvolver um pouco a álgebra até que se chegue numa equação no formato da definição).

\section{Variações e taxas de variação}

Usamos o símbolo "$\Delta$"\footnote{O nome é "delta".} para denotar a variação de alguma variável. Ou seja, se tivemos uma variável qualquer $x$ que teve seu valor alterado de $x^1$ para $x^2$, então:

$$ \Delta x = x^2 - x^1 $$
ou também
$$ x^2 = x^1 + \Delta x $$
\\
Normalmente, usamos o delta quando falamos de \textbf{pequenas variações} ou, como os economistas falam, \textbf{variações marginais}.
\\
\\
A \textbf{taxa de variação} é obtida pela razão (ou seja, pela divisão) de duas variações. Seja a função $y = f(x)$, sempre que tivemos um $\Delta x > 0$ também teremos algum $\Delta y \neq 0$. A taxa de variação de $y$ em relação à $x$ é dada por:

$$ \frac{\Delta y}{\Delta x} = \frac{y^2 - y^1}{x^2 - x^1} = \frac{f(x^1 + \Delta x) - f(x^1)}{\Delta x} $$
\\
É uma medida do quanto $y$ varia a medida que $x$ varia.
\\
\\
Quando uma função é linear, teremos que essa taxa de variação será sempre constante para quaisquer valores de $x$. Como $y = ax + b$, então
\\
\\
\Large $ \frac{\Delta y}{\Delta x} = $ \normalsize
$$ \frac{a+b(x^1 + \Delta x) - (a + bx^1)}{\Delta x} = $$
$$ \frac{\cancel{a}+b(x^1 + \Delta x) \cancel{-a} - bx^1)}{\Delta x} = $$
$$ \frac{\cancel{bx^1} + b \Delta x \cancel{- bx^1}}{\Delta x} = $$
$$ \frac{b \cancel{\Delta x}}{\cancel{\Delta x}} = b  $$

Para as funções não lineares, essa propriedade não é observada. Tomemos $y = f(x) = x^2$ como exemplo,
\\
\\
\Large $ \frac{\Delta y}{\Delta x} = $ \normalsize
$$ \frac{(x + \Delta x)^2 - x^2}{\Delta x} = $$ 
$$  \frac{\cancel{x^2} + 2x \Delta x + (\Delta x)^2 \cancel{-x^2}}{\Delta x} = $$
$$  \frac{2x \cancel{\Delta x} + \Delta x . \cancel{\Delta x}}{\cancel{\Delta x}} = $$
$$  2x + \Delta x $$
\\
Ou seja, entra no resultado da taxa de variação o valor de $x$ e a magnitude da variação, dada por $\Delta x$.

\section{Inclinações e interceptos}

Já aprendemos como calcular a taxa de variação de uma função. Graficamente falando, essa é a medida da inclinação da curva da função entre os dois pontos que formam o delta da variável independente. 
\\
\\
Em uma função linear, a inclinação da curva sempre será a mesma independente da magnitude da variação. No caso das funções não lineares, a inclinação é dada pela \textbf{reta tangente} ao ponto da curva\footnote{Mais pra frente a gente volta nessa ideia.}.
\\
\\
No caso de uma função linear, $ y = ax + b$, temos alguns pontos que recebem nomes de \textbf{intercepto}. O \textbf{intercepto vertical} ($y^*$) é dado pelo ponto $y = a.0 + b = b$, ou seja, onde $x = 0$. Já o \textbf{intercepto horizontal} ($x^*$) é dado pelo ponto onde $y = ax + b = 0 $, ou seja, $ x = \frac{-b}{a}$.

\section{Valores absolutos e logaritmos}

O \textbf{valor absoluto} de um número $x$ qualquer é definido pela função $f(x)$ do seguinte modo:

\[ f(x) = |x| = \begin{cases} x & se \ x \geqslant \\ -x & se \ x < 0 \end{cases} \]
\\
\\
Você já deve ter visto no ensino médio que o \textbf{logaritmo natural} ou \textbf{log} de um número é uma função escrita como $y = lnx$ ou $y = ln(x)$ e que possui as seguintes propriedades:

\begin{itemize}
 \item Se $x,y > 0$, então, $ ln(xy) = ln(x) + ln(y) $
 \item $ ln(e) = 1 $
 \item $ ln(x^y) = y ln(x) $
\end{itemize}

\section{Derivadas}
\section{Derivadas segundas}
\section{A regra do produto e da cadeia}
\section{Derivadas parciais}

Nós já vimos no ponto 1.1 que funções podem conter mais de uma variável independente. Supondo uma função composta $f(x_1,x_2)$ a sua \textbf{derivada parcial} em relação a $x_1$ será dada por:

$$ \frac{\partial f(x_1,x_2)}{\partial x_1} = 
\lim_{\Delta x_1 \to 0} \frac{f(x_1+\Delta x_1,x_2) - f(x_1,x_2)}{\Delta x_1} $$
\\
similarmente, a derivada parcial em relação a $x_2$ será dada por
\\
$$ \frac{\partial f(x_1,x_2)}{\partial x_2} = 
\lim_{\Delta x_2 \to 0} \frac{f(x_1,x_2+\Delta x_2) - f(x_1,x_2)}{\Delta x_2} $$
\\
A ideia por trás de uma derivada parcial é verificar a taxa de variação entre a nossa função composta em relação a alguma variação de apenas uma das variáveis independentes, ou seja, é como se tratássemos as outras variáveis como constantes.
\\
\\
As propriedades das derivadas parciais são parecidas com as normais. Exceção é a regra da cadeia. Seja a função composta $g(t) = f(x_1(t),x_2(t))$, então a derivada de $g(t)$ em relação a $t$ é dada por:

$$ \frac{dg(t)}{dt} = 
\frac{\partial f(x_1,x_2)}{\partial x_1}\frac{dx_1(t)}{dt} + 
\frac{\partial f(x_1,x_2)}{\partial x_2}\frac{dx_2(t)}{dt} $$
\\
Atente para o fato que as variáveis independentes da nossa função $g(t)$ são as funções $x_1(t)$ e $x_2(t)$ que também têm como variável independente $t$.

\section{Otimização}

A maioria dos modelos utilizados pela Economia podem ser expressos como um problemas de otimização. Matematicamente falando, dada uma função $y = f(x)$ seu valor \textbf{máximo} será dado ponto $x^*$ se $f(x^*) \geqslant f(x)$ para qualquer valor de $x$. Não faz parte do escopo desse apêndice demonstrar isso, então tenha fé que, se uma função for suave, o seu valor máximo é obtido no ponto onde teremos

$$ \frac{df(x^*)}{dx} = 0 $$

e também

$$ \frac{d^2f(x^*)}{dx^2} \leq 0$$
\\
Ou seja, o máximo será o ponto onde a derivada for igual a zero e a derivada segunda for menor igual a zero. Chamamos a primeira de \textbf{condição de primeira ordem} e a segunda de \textbf{condição de segunda ordem}.
\\
\\
Também é muito comum buscarmos a minimização de determinadas funções. Nesse caso, só teremos uma pequena mudança na condição de segunda ordem;

$$ \frac{df(x^*)}{dx} = 0 $$

e também

$$ \frac{d^2f(x^*)}{dx^2} \geq 0$$
\\
No casos das funções compostas suaves, as condições de primeira ordem para os pontos de máximo e mínimo são alcançadas no ponto $(x_{1}^*,x_{2}^*)$ cujas derivadas serão

$$ \frac{\partial f(x_{1}^*,x_{2}^*)}{\partial x_1} = 0 $$
e
$$ \frac{\partial f(x_{1}^*,x_{2}^*)}{\partial x_2} = 0 $$
\\
As condições de segunda ordem são muito mais complexas então não fazem parte do escopo desse curso.

\section{Otimização com restrição}

Saber maximizar ou minimizar uma função é só uma parte do problema de otimização. Na vida real, a esmagadora maioria das situações de otimização está contida dentro de algum limite de possibilidades. A \textbf{otimização com restrição} é a técnica usada para encontrar o ponto de máximo ou mínimo de alguma função dentro de um determinado domínio de possibilidades.

\begin{center}
\LARGE $\stackrel{máx}{\text{\small $x_1,x_2$}} \ \ \stackrel{f(x_1,x_2)}{\ }$ \\
\normalsize $\textrm{de modo que } g(x_1,x_2) = c$
\end{center}

A função $f(x_1,x_2)$ é chamada de \textbf{função objeto} e a equação $g(x_1,x_2) = c$ é chamada de \textbf{restrição}.

%%%%%%%%%%%%%%%%%%%%%%%% CHAPTER %%%%%%%%%%%%%%%%%%%%%%%%
\chapter{Programação}

Ao longo do livro eu vou construir alguns programas em Python para simular alguns modelos. Não faz parte do escopo desse livro ensinar como fazer isso. Contudo, eu posso manter aqui os links dos arquivos usados nos modelos para que possam ser estudados por contra própria.\footnote{Ou talvez a gente faça um curso sobre isso.}


%%%%%%%%%%%%%%%%%%%%%%%% PART %%%%%%%%%%%%%%%%%%%%%%%%
\part{Teoria da Escolha}

%%%%%%%%%%%%%%%%%%%%%%%% CHAPTER %%%%%%%%%%%%%%%%%%%%%%%%
\chapter{O Mercado}

\begin{chapquote}{página 3}
	``The theory of sets is a language that is perfectly suited to describing and explaning all types of mathematical structures.''
\end{chapquote}


%\section{A elaboração de um modelo}
%\section{Otimização e equilíbrio}
%\section{A curva de demanda}
%\section{A curva de oferta}
%\section{O equilíbrio de mercado}
%\section{A estática comparativa}
%\section{Outras formas de alocar apartamentos}
%\section{Qual o melhor arranjo?}
%\section{A eficiência de Pareto}
%\section{Comparação entra as formas de alocação de apartamentos}
%\section{Equilíbrio no longo prazo}

%%%%%%%%%%%%%%%%%%%%%%%% CHAPTER %%%%%%%%%%%%%%%%%%%%%%%%
\chapter{Restrição Orçamentária}

%%%%%%%%%%%%%%%%%%%%%%%% CHAPTER %%%%%%%%%%%%%%%%%%%%%%%%
\chapter{Preferências}

%%%%%%%%%%%%%%%%%%%%%%%% CHAPTER %%%%%%%%%%%%%%%%%%%%%%%%
\chapter{Utilidade}

%%%%%%%%%%%%%%%%%%%%%%%% CHAPTER %%%%%%%%%%%%%%%%%%%%%%%%
\chapter{Escolha}

%%%%%%%%%%%%%%%%%%%%%%%% CHAPTER %%%%%%%%%%%%%%%%%%%%%%%%
\chapter{Demanda}

%%%%%%%%%%%%%%%%%%%%%%%% CHAPTER %%%%%%%%%%%%%%%%%%%%%%%%
\chapter{Preferência Revelada}

%%%%%%%%%%%%%%%%%%%%%%%% CHAPTER %%%%%%%%%%%%%%%%%%%%%%%%
\chapter{A Equação de Slutsky}

%%%%%%%%%%%%%%%%%%%%%%%% CHAPTER %%%%%%%%%%%%%%%%%%%%%%%%
\chapter{Restrição Orçamentária}

%%%%%%%%%%%%%%%%%%%%%%%% CHAPTER %%%%%%%%%%%%%%%%%%%%%%%%
\chapter{Comprando e Vendendo}

%%%%%%%%%%%%%%%%%%%%%%%% CHAPTER %%%%%%%%%%%%%%%%%%%%%%%%
\chapter{Escolha Intertermporal}

%%%%%%%%%%%%%%%%%%%%%%%% CHAPTER %%%%%%%%%%%%%%%%%%%%%%%%
\chapter{Mercado de Ativos}

%%%%%%%%%%%%%%%%%%%%%%%% CHAPTER %%%%%%%%%%%%%%%%%%%%%%%%
\chapter{Incerteza}

%%%%%%%%%%%%%%%%%%%%%%%% CHAPTER %%%%%%%%%%%%%%%%%%%%%%%%
\chapter{Ativos de Risco}

%%%%%%%%%%%%%%%%%%%%%%%% CHAPTER %%%%%%%%%%%%%%%%%%%%%%%%
\chapter{O Excedente do Consumidor}

%%%%%%%%%%%%%%%%%%%%%%%% CHAPTER %%%%%%%%%%%%%%%%%%%%%%%%
\chapter{Demanda de Mercado}

%%%%%%%%%%%%%%%%%%%%%%%% PART %%%%%%%%%%%%%%%%%%%%%%%%
\part{Equilíbrio, Econometria e Leilões}

%%%%%%%%%%%%%%%%%%%%%%%% CHAPTER %%%%%%%%%%%%%%%%%%%%%%%%
\chapter{Equilíbrio}

%%%%%%%%%%%%%%%%%%%%%%%% CHAPTER %%%%%%%%%%%%%%%%%%%%%%%%
\chapter{Medição}

%%%%%%%%%%%%%%%%%%%%%%%% CHAPTER %%%%%%%%%%%%%%%%%%%%%%%%
\chapter{Leilões}

%%%%%%%%%%%%%%%%%%%%%%%% CHAPTER %%%%%%%%%%%%%%%%%%%%%%%%
\chapter{Equilíbrio}

%%%%%%%%%%%%%%%%%%%%%%%% PART %%%%%%%%%%%%%%%%%%%%%%%%
\part{Teoria da Firma}

%%%%%%%%%%%%%%%%%%%%%%%% CHAPTER %%%%%%%%%%%%%%%%%%%%%%%%
\chapter{Tecnologia}

%%%%%%%%%%%%%%%%%%%%%%%% CHAPTER %%%%%%%%%%%%%%%%%%%%%%%%
\chapter{Maximização do Lucro}

%%%%%%%%%%%%%%%%%%%%%%%% CHAPTER %%%%%%%%%%%%%%%%%%%%%%%%
\chapter{Minimização de Custos}

%%%%%%%%%%%%%%%%%%%%%%%% CHAPTER %%%%%%%%%%%%%%%%%%%%%%%%
\chapter{Curva de Custo}

%%%%%%%%%%%%%%%%%%%%%%%% CHAPTER %%%%%%%%%%%%%%%%%%%%%%%%
\chapter{Oferta da Empresa}

%%%%%%%%%%%%%%%%%%%%%%%% CHAPTER %%%%%%%%%%%%%%%%%%%%%%%%
\chapter{Oferta da Indústria}

%%%%%%%%%%%%%%%%%%%%%%%% PART %%%%%%%%%%%%%%%%%%%%%%%%
\part{Mercados}

%%%%%%%%%%%%%%%%%%%%%%%% CHAPTER %%%%%%%%%%%%%%%%%%%%%%%%
\chapter{Monopólio}

Anteriormente, fora demonstrado como a oferta pode ser construída da firma individual até a indústria competitiva. Nesse cenário, todos os ofertantes não possuem poder de interferir no preço e na quantidade de equilíbrio do mercado. Mas podemos pensar num caso muito diferente: Como seria o caso onde só exista uma empresa que controle toda a oferta?
\\
\\
Diferente dos casos anteriores, precisamos construir um modelo de tomada de decisão que leve em consideração a capacidade do monopolista de intervir diretamente no preço de modo a maximizar seus lucros totais.
\\
\\
Existem duas maneiras de enxergar esse problema. Podemos modelar como se o monopolista controlasse o preço e a demanda é quem definiria a quantidade. Ou, ao contrário, podemos modelar como se o monopolista definisse a quantidade a ser produzida e a demanda definiria o seu preço de equilíbrio para essa quantidade.
\\
\\
Independente do modelo, podemos ver que as abordagens são equivalentes. Por facilidade analítica vamos seguir a abordagem de definição da quantidade produzida.

\section{Maximização dos Lucros}

Como vimos nos capítulo 01, estamos diante de um problema de maximização. Sendo mais preciso, nós queremos maximizar o lucro do monopolista dado por $yp(y) - c(y)$ onde $p(y)$ é a demanda inversa\footnote{Ou seja, ela mostra o preço de equilíbrio dada uma quantidade qualquer.} para o mercado, $r(y) = yp(y)$ é a receita do monopolista e $c(y)$ é o custo de produção das $y$ unidades. Podemos resumir nosso problema como

\begin{center}
\LARGE $\stackrel{máx}{\text{\small $y$}} \ \ \stackrel{r(y) - c(y)}{\ }$ \\
\end{center}

A condição de otimização é evidente: A receita marginal deve ser igual ao custo marginal. Se a receita marginal for maior, bastaria aumentar a produção para aumentar os lucros. Se fosse menor, seria necessário reduzir a quantidade produzida afim de elevar o preço a um nível satisfatório. Algebricamente, temos que

$$ \textrm{RM = CMa} $$
$$ ou $$
$$ \frac{\Delta r}{\Delta y} = \frac{\Delta c}{\Delta y} $$
\\
Até aqui a gente tá bem perto da modelagem para as firmas competidoras. O custo marginal é definido pela tecnologia de produção. A mudança acontecerá na receita marginal. 
\\
\\
Como o monopolista tem o poder de intervir no mercado, sempre que ele decidir alterar a produção em $\Delta y$ unidades, haverá dois efeitos na receita. Em primeiro lugar, ele terá um aumento na receita em $p\Delta y$ unidades. Em segundo lugar, como o mercado terá mais bens a sua disposição, ele estará disposto a pagar um preço menor pelas novas unidades, ou seja, $y\Delta p$. O resultado líquido desse efeito é obtido por

$$\Delta r = p \Delta y + y \Delta p$$

$$ \frac{\Delta r}{\Delta y} = 
\frac{p \Delta y}{\Delta y} + 
\frac{y \Delta p}{\Delta y} $$

$$ \frac{\Delta r}{\Delta y} = 
\frac{p \cancel{\Delta y}}{\cancel{\Delta y}} + 
\frac{y \Delta p}{\Delta y} $$

$$ \frac{\Delta r}{\Delta y} = p + \frac{y \Delta p}{\Delta y} $$

\section{Curva de Demanda Linear e Monopólio}
\section{Estabelecimento de Preços com Markup}
\section{A Ineficiência do Monopólio}
\section{O Ônus do Monopólio}
\section{Monopólio Natural}
\section{O Que Causa os Monopólios?}

%%%%%%%%%%%%%%%%%%%%%%%% CHAPTER %%%%%%%%%%%%%%%%%%%%%%%%
\chapter{O Comportamento do Monipolista}

%%%%%%%%%%%%%%%%%%%%%%%% CHAPTER %%%%%%%%%%%%%%%%%%%%%%%%
\chapter{O Mercado de Fatores}

%%%%%%%%%%%%%%%%%%%%%%%% CHAPTER %%%%%%%%%%%%%%%%%%%%%%%%
\chapter{O Oligopólio}

%%%%%%%%%%%%%%%%%%%%%%%% CHAPTER %%%%%%%%%%%%%%%%%%%%%%%%
\chapter{A Teoria dos Jogos}

%%%%%%%%%%%%%%%%%%%%%%%% CHAPTER %%%%%%%%%%%%%%%%%%%%%%%%
\chapter{Aplicações da Teoria dos Jogos}

%%%%%%%%%%%%%%%%%%%%%%%% PART %%%%%%%%%%%%%%%%%%%%%%%%
\part{Tópicos Avançados}

%%%%%%%%%%%%%%%%%%%%%%%% CHAPTER %%%%%%%%%%%%%%%%%%%%%%%%
\chapter{Economia Comportamental}

%%%%%%%%%%%%%%%%%%%%%%%% CHAPTER %%%%%%%%%%%%%%%%%%%%%%%%
\chapter{Trocas}

%%%%%%%%%%%%%%%%%%%%%%%% CHAPTER %%%%%%%%%%%%%%%%%%%%%%%%
\chapter{Produção}

%%%%%%%%%%%%%%%%%%%%%%%% CHAPTER %%%%%%%%%%%%%%%%%%%%%%%%
\chapter{O Bem-Estar}

%%%%%%%%%%%%%%%%%%%%%%%% CHAPTER %%%%%%%%%%%%%%%%%%%%%%%%
\chapter{Externalidades}

%%%%%%%%%%%%%%%%%%%%%%%% CHAPTER %%%%%%%%%%%%%%%%%%%%%%%%
\chapter{Tecnologia da Informação}

%%%%%%%%%%%%%%%%%%%%%%%% CHAPTER %%%%%%%%%%%%%%%%%%%%%%%%
\chapter{Bens Públicos}

%%%%%%%%%%%%%%%%%%%%%%%% CHAPTER %%%%%%%%%%%%%%%%%%%%%%%%
\chapter{Informação Assimétrica}

\end{document}
