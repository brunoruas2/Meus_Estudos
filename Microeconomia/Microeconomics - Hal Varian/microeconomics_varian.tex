%%%%%%%%%%%%%%%%%%%%%%%%%%%%%%%%%%%%%%%%%%%%%%%%%%%
%% LaTeX book template                           %%
%% Author:  Amber Jain (http://amberj.devio.us/) %%
%% License: ISC license                          %%
%%%%%%%%%%%%%%%%%%%%%%%%%%%%%%%%%%%%%%%%%%%%%%%%%%%

\documentclass[a4paper,11pt,oneside]{book}

% pacotes utilizados
\usepackage[T1]{fontenc}
\usepackage[utf8]{inputenc}
\usepackage{lmodern}
\usepackage{hyperref}
\usepackage{graphicx}
\usepackage[portuguese]{babel}
\usepackage{amsfonts}
\usepackage[usenames,dvipsnames]{xcolor}
\usepackage{mathtools}
\usepackage{amssymb} % alguns simbolos matematicos
\usepackage{mathrsfs} % letras cursivas
\usepackage{listings} % coding examples in latex file
\usepackage{xcolor} % changing colors
\usepackage{amsthm} % definir estilo do teorema
\usepackage[hang,flushmargin]{footmisc} % remove footnote's identation
\usepackage{cancel} % para poder colocar o tracinho de cancelamento
\usepackage{tikz} % to draw Venn's diagrams
\usepackage{amsmath} % to write a function by cases

% dark theme no pdf
\pagecolor[rgb]{0.1,0.1,0.1} %black
\color[rgb]{0.9,0.9,0.9} %grey

% criando o modelo de definicoes/teoremas/fatos/demonstracoes
\theoremstyle{definition}

\newtheoremstyle{break}% name
  {10pt}%         Space above, empty = `usual value'
  {10pt}%         Space below
  {}% Body font
  {}%         Indent amount (empty = no indent, \parindent = para indent)
  {\bfseries}% Thm head font
  {}%        Punctuation after thm head
  {\newline}% Space after thm head: \newline = linebreak
  {}%         Thm head spec

\theoremstyle{break}

% definindo as categorias de formalidade
\newtheorem{definition}{Definição}[section]
\newtheorem{fact}{Fato}[section]
\newtheorem{demonstration}{Demonstração}[section]
\newtheorem{theorem}{Teorema}

% tirando identação dos paragrafos
\setlength{\parindent}{0ex}

% setting das cores quando usar codigo python
\definecolor{codegreen}{rgb}{0,0.6,0}
\definecolor{codegray}{rgb}{0.5,0.5,0.5}
\definecolor{codepurple}{rgb}{0.58,0,0.82}
\definecolor{backcolour}{rgb}{0.95,0.95,0.92}

\lstdefinestyle{mystyle}{
    backgroundcolor=\color{backcolour},   
    commentstyle=\color{codegreen},
    keywordstyle=\color{magenta},
    numberstyle=\tiny\color{codegray},
    stringstyle=\color{codepurple},
    basicstyle=\ttfamily\footnotesize,
    breakatwhitespace=false,         
    breaklines=true,                 
    captionpos=b,                    
    keepspaces=true,                 
    numbers=left,                    
    numbersep=5pt,                  
    showspaces=false,                
    showstringspaces=false,
    showtabs=false,                  
    tabsize=2
}

\lstset{style=mystyle}

% dedicatoria 
% Source: http://www.tug.org/pipermail/texhax/2010-June/
\newenvironment{dedication}
{
   \cleardoublepage
   \thispagestyle{empty}
   \vspace*{\stretch{1}}
   \hfill\begin{minipage}[t]{0.66\textwidth}
   \raggedright
}
{
   \end{minipage}
   \vspace*{\stretch{3}}
   \clearpage
}

% Chapter quote at the start of chapter        %
% Source: http://tex.stackexchange.com/a/53380 %
\makeatletter
\renewcommand{\@chapapp}{}% Not necessary...
\newenvironment{chapquote}[2][2em]
  {\setlength{\@tempdima}{#1}%
   \def\chapquote@author{#2}%
   \parshape 1 \@tempdima \dimexpr\textwidth-2\@tempdima\relax%
   \itshape}
  {\par\normalfont\hfill--\ \chapquote@author\hspace*{\@tempdima}\par\bigskip}
\makeatother


%%%%%%%%%%%%%%%%%%%%%%%%%%%%%%%%%%%%%%%%%%%%%%%%%%%
% First page of book which contains 'stuff' like: %
%  - Book title, subtitle                         %
%  - Book author name                             %
%%%%%%%%%%%%%%%%%%%%%%%%%%%%%%%%%%%%%%%%%%%%%%%%%%%
% Book's title and subtitle
\title{\Huge \textbf{Microeconomia} \\ 
\Large Tradução da 9 edição \\
\huge by Hal R. Varian}

% Author
\author{
\textsc{Resumo e Adaptação por:} \\
\textsc{Bruno de M. Ruas}
}

\begin{document}

\frontmatter
\maketitle

\tableofcontents
%\listoffigures
%\listoftables

\mainmatter

%%%%%%%%%%%%%%%%%%%%%%%% PART %%%%%%%%%%%%%%%%%%%%%%%%
\part{Preparativos}

%%%%%%%%%%%%%%%%%%%%%%%% CHAPTER %%%%%%%%%%%%%%%%%%%%%%%%
\chapter{Matemática}

\begin{chapquote}{página 1.008}
	``Revisão breve de alguns conceitos matemáticos utilizados no texto''.
\end{chapquote}

Bem vindo ao meu resumo do livro do prof. Varian. Ao contrário do que ele fez, eu preferi trazer o apêndice de matemática pro começo do material porque aqui nós vamos ver as ferramentas que serão usadas para a explicação dos conceitos teóricos ao longo do material.
\\
\\
Aqui a gente só vai dar um overview básico nos conceitos. Não tenha dúvida que alguém mais experimentado em matemática torceria o nariz pra algumas definições dadas aqui. Mas o objetivo é te dar um "norte"\ a respeito de alguns conceitos normalmente usados. Não se assuste com a simplicidade de algumas coisas. Melhor garantir agora do que sofrer mais pra frente no texto.

\section{Funções}

Sejam dois números quaisquer $x$ e $y$, uma \textbf{função} ou \textbf{transformação} é uma regra que descreve uma relação entre eles.
\\
\\
Para demonstrar que existe alguma dependência entre duas variáveis usamos a notação $y = f(x)$, onde nossa variável $y$ (chamada de \textbf{dependente}) é o resultado de alguma transformação (denotada pelo símbolo $"f"$) realizada em $x$ (nossa variável \textbf{independente}).
\\
\\
Não é raro ter uma variável dependente relacionada a várias outras variáveis. Nesses casos é comum o uso da notação anterior com a adição das novas incógnitas. Algo como $y = f(x_1,x_2,...,x_n)$.

\section{Gráficos}

Não tem muito o que falar aqui. Dá uma lida lá na página 1010.

\section{Propriedades de funções}

Uma função pode ter algumas características que facilitam a sua descrição. Aqui temos algumas que serão usadas ao longo do curso:
\\
\\
Uma \textbf{função contínua} é aquela que não possui nenhum "salto"\ ou "quebra". 
\\
\\
Uma \textbf{função suave} é aquela que não tem "dobras"\ nem "cantos".
\\
\\
Uma \textbf{função monotônica} é aquela que sempre segue o mesmo sentido (ou crescendo ou decrescendo) sem nunca mudar de sentido. 
Quando é crescente a medida que $x$ cresce, chamaremos de \textbf{função monotônica crescente}. Quando descrescer a medida que $x$ crescer, chamaremos de \textbf{função monotônica decrescente}.

\section{Funções inversas}

Uma das implicações de quando uma função é monotônica é que, para cada $x$, sempre existirá apenas um único $y$ associado. 
\\
\\
Uma \textbf{função inversa} é a função que, sempre que colocarmos um $y$ como variável independente teremos como resultado um $x$ de alguma função anterior.\footnote{Eu tentei não deixar confuso mas se ficou com dúvida, pesquisa um pouco sobre o tema.}

\section{Equações e identidades}

Podemos relacionar dois ou mais elementos por meio do uso de \textbf{equações} (usando o símbolo da igualdade "$=$"). Onde as suas respectivas \textbf{soluções} são os valores atribuíveis as incógnitas que assegurem a validade da relação proposta.
\\
\\
Uma \textbf{identidade} (que tem o símbolo dado por "$\equiv$") é um tipo de relação onde sempre haverá as soluções independentemente de quais valores suas variáveis assumam.

\section{Funções lineares}

Chamamos de \textbf{função linear}, qualquer função da forma $y = ax + b$. Fique atento porque uma função linear pode ser expressa de maneira implícita (ou seja, será necessário desenvolver um pouco a álgebra até que se chegue numa equação no formato da definição).

\section{Variações e taxas de variação}

Usamos o símbolo "$\Delta$"\footnote{O nome é "delta".} para denotar a variação de alguma variável. Ou seja, se tivemos uma variável qualquer $x$ que teve seu valor alterado de $x^1$ para $x^2$, então:

$$ \Delta x = x^2 - x^1 $$
ou também
$$ x^2 = x^1 + \Delta x $$
\\
Normalmente, usamos o delta quando falamos de \textbf{pequenas variações} ou, como os economistas falam, \textbf{variações marginais}.
\\
\\
A \textbf{taxa de variação} é obtida pela razão (ou seja, pela divisão) de duas variações. Seja a função $y = f(x)$, sempre que tivemos um $\Delta x > 0$ também teremos algum $\Delta y \neq 0$. A taxa de variação de $y$ em relação à $x$ é dada por:

$$ \frac{\Delta y}{\Delta x} = \frac{y^2 - y^1}{x^2 - x^1} = \frac{f(x^1 + \Delta x) - f(x^1)}{\Delta x} $$
\\
É uma medida do quanto $y$ varia a medida que $x$ varia.
\\
\\
Quando uma função é linear, teremos que essa taxa de variação será sempre constante para quaisquer valores de $x$. Como $y = ax + b$, então
\\
\\
\Large $ \frac{\Delta y}{\Delta x} = $ \normalsize
$$ \frac{a+b(x^1 + \Delta x) - (a + bx^1)}{\Delta x} = $$
$$ \frac{\cancel{a}+b(x^1 + \Delta x) \cancel{-a} - bx^1)}{\Delta x} = $$
$$ \frac{\cancel{bx^1} + b \Delta x \cancel{- bx^1}}{\Delta x} = $$
$$ \frac{b \cancel{\Delta x}}{\cancel{\Delta x}} = b  $$

Para as funções não lineares, essa propriedade não é observada. Tomemos $y = f(x) = x^2$ como exemplo,
\\
\\
\Large $ \frac{\Delta y}{\Delta x} = $ \normalsize
$$ \frac{(x + \Delta x)^2 - x^2}{\Delta x} = $$ 
$$  \frac{\cancel{x^2} + 2x \Delta x + (\Delta x)^2 \cancel{-x^2}}{\Delta x} = $$
$$  \frac{2x \cancel{\Delta x} + \Delta x . \cancel{\Delta x}}{\cancel{\Delta x}} = $$
$$  2x + \Delta x $$
\\
Ou seja, entra no resultado da taxa de variação o valor de $x$ e a magnitude da variação, dada por $\Delta x$.

\section{Inclinações e interceptos}

Já aprendemos como calcular a taxa de variação de uma função. Graficamente falando, essa é a medida da inclinação da curva da função entre os dois pontos que formam o delta da variável independente. 
\\
\\
Em uma função linear, a inclinação da curva sempre será a mesma independente da magnitude da variação. No caso das funções não lineares, a inclinação é dada pela \textbf{reta tangente} ao ponto da curva\footnote{Mais pra frente a gente volta nessa ideia.}.
\\
\\
No caso de uma função linear, $ y = ax + b$, temos alguns pontos que recebem nomes de \textbf{intercepto}. O \textbf{intercepto vertical} ($y^*$) é dado pelo ponto $y = a.0 + b = b$, ou seja, onde $x = 0$. Já o \textbf{intercepto horizontal} ($x^*$) é dado pelo ponto onde $y = ax + b = 0 $, ou seja, $ x = \frac{-b}{a}$.

\section{Valores absolutos e logaritmos}

O \textbf{valor absoluto} de um número $x$ qualquer é definido pela função $f(x)$ do seguinte modo:

\[ f(x) = |x| = \begin{cases} x & se \ x \geqslant \\ -x & se \ x < 0 \end{cases} \]
\\
\\
Você já deve ter visto no ensino médio que o \textbf{logaritmo natural} ou \textbf{log} de um número é uma função escrita como $y = lnx$ ou $y = ln(x)$ e que possui as seguintes propriedades:

\begin{itemize}
 \item Se $x,y > 0$, então, $ ln(xy) = ln(x) + ln(y) $
 \item $ ln(e) = 1 $
 \item $ ln(x^y) = y ln(x) $
\end{itemize}

\section{Derivadas}
\section{Derivadas segundas}
\section{A regra do produto e da cadeia}
\section{Derivadas parciais}
\section{Otimização}
\section{Otimização com restrição}


%%%%%%%%%%%%%%%%%%%%%%%% CHAPTER %%%%%%%%%%%%%%%%%%%%%%%%
\chapter{Programação}

%%%%%%%%%%%%%%%%%%%%%%%% PART %%%%%%%%%%%%%%%%%%%%%%%%
\part{Teoria da Escolha}

%%%%%%%%%%%%%%%%%%%%%%%% CHAPTER %%%%%%%%%%%%%%%%%%%%%%%%
\chapter{O Mercado}

\begin{chapquote}{página 3}
	``The theory of sets is a language that is perfectly suited to describing and explaning all types of mathematical structures.''
\end{chapquote}

\section{A elaboração de um modelo}
\section{Otimização e equilíbrio}
\section{A curva de demanda}
\section{A curva de oferta}
\section{O equilíbrio de mercado}
\section{A estática comparativa}
\section{Outras formas de alocar apartamentos}
\section{Qual o melhor arranjo?}
\section{A eficiência de Pareto}
\section{Comparação entra as formas de alocação de apartamentos}
\section{Equilíbrio no longo prazo}

%%%%%%%%%%%%%%%%%%%%%%%% CHAPTER %%%%%%%%%%%%%%%%%%%%%%%%
\chapter{Restrição Orçamentária}

%%%%%%%%%%%%%%%%%%%%%%%% CHAPTER %%%%%%%%%%%%%%%%%%%%%%%%
\chapter{Preferências}

%%%%%%%%%%%%%%%%%%%%%%%% CHAPTER %%%%%%%%%%%%%%%%%%%%%%%%
\chapter{Utilidade}

%%%%%%%%%%%%%%%%%%%%%%%% CHAPTER %%%%%%%%%%%%%%%%%%%%%%%%
\chapter{Escolha}

%%%%%%%%%%%%%%%%%%%%%%%% CHAPTER %%%%%%%%%%%%%%%%%%%%%%%%
\chapter{Demanda}

%%%%%%%%%%%%%%%%%%%%%%%% CHAPTER %%%%%%%%%%%%%%%%%%%%%%%%
\chapter{Preferência Revelada}

%%%%%%%%%%%%%%%%%%%%%%%% CHAPTER %%%%%%%%%%%%%%%%%%%%%%%%
\chapter{A Equação de Slutsky}

%%%%%%%%%%%%%%%%%%%%%%%% CHAPTER %%%%%%%%%%%%%%%%%%%%%%%%
\chapter{Restrição Orçamentária}

%%%%%%%%%%%%%%%%%%%%%%%% CHAPTER %%%%%%%%%%%%%%%%%%%%%%%%
\chapter{Comprando e Vendendo}

%%%%%%%%%%%%%%%%%%%%%%%% CHAPTER %%%%%%%%%%%%%%%%%%%%%%%%
\chapter{Escolha Intertermporal}

%%%%%%%%%%%%%%%%%%%%%%%% CHAPTER %%%%%%%%%%%%%%%%%%%%%%%%
\chapter{Mercado de Ativos}

%%%%%%%%%%%%%%%%%%%%%%%% CHAPTER %%%%%%%%%%%%%%%%%%%%%%%%
\chapter{Incerteza}

%%%%%%%%%%%%%%%%%%%%%%%% CHAPTER %%%%%%%%%%%%%%%%%%%%%%%%
\chapter{Ativos de Risco}

%%%%%%%%%%%%%%%%%%%%%%%% CHAPTER %%%%%%%%%%%%%%%%%%%%%%%%
\chapter{O Excedente do Consumidor}

%%%%%%%%%%%%%%%%%%%%%%%% CHAPTER %%%%%%%%%%%%%%%%%%%%%%%%
\chapter{Demanda de Mercado}

%%%%%%%%%%%%%%%%%%%%%%%% PART %%%%%%%%%%%%%%%%%%%%%%%%
\part{Equilíbrio, Econometria e Leilões}

%%%%%%%%%%%%%%%%%%%%%%%% CHAPTER %%%%%%%%%%%%%%%%%%%%%%%%
\chapter{Equilíbrio}

%%%%%%%%%%%%%%%%%%%%%%%% CHAPTER %%%%%%%%%%%%%%%%%%%%%%%%
\chapter{Medição}

%%%%%%%%%%%%%%%%%%%%%%%% CHAPTER %%%%%%%%%%%%%%%%%%%%%%%%
\chapter{Leilões}

%%%%%%%%%%%%%%%%%%%%%%%% CHAPTER %%%%%%%%%%%%%%%%%%%%%%%%
\chapter{Equilíbrio}

%%%%%%%%%%%%%%%%%%%%%%%% PART %%%%%%%%%%%%%%%%%%%%%%%%
\part{Teoria da Firma}

%%%%%%%%%%%%%%%%%%%%%%%% CHAPTER %%%%%%%%%%%%%%%%%%%%%%%%
\chapter{Tecnologia}

%%%%%%%%%%%%%%%%%%%%%%%% CHAPTER %%%%%%%%%%%%%%%%%%%%%%%%
\chapter{Maximização do Lucro}

%%%%%%%%%%%%%%%%%%%%%%%% CHAPTER %%%%%%%%%%%%%%%%%%%%%%%%
\chapter{Minimização de Custos}

%%%%%%%%%%%%%%%%%%%%%%%% CHAPTER %%%%%%%%%%%%%%%%%%%%%%%%
\chapter{Curva de Custo}

%%%%%%%%%%%%%%%%%%%%%%%% CHAPTER %%%%%%%%%%%%%%%%%%%%%%%%
\chapter{Oferta da Empresa}

%%%%%%%%%%%%%%%%%%%%%%%% CHAPTER %%%%%%%%%%%%%%%%%%%%%%%%
\chapter{Oferta da Indústria}

%%%%%%%%%%%%%%%%%%%%%%%% PART %%%%%%%%%%%%%%%%%%%%%%%%
\part{Mercados}

%%%%%%%%%%%%%%%%%%%%%%%% CHAPTER %%%%%%%%%%%%%%%%%%%%%%%%
\chapter{Monopólio}

%%%%%%%%%%%%%%%%%%%%%%%% CHAPTER %%%%%%%%%%%%%%%%%%%%%%%%
\chapter{O Comportamento do Monipolista}

%%%%%%%%%%%%%%%%%%%%%%%% CHAPTER %%%%%%%%%%%%%%%%%%%%%%%%
\chapter{O Mercado de Fatores}

%%%%%%%%%%%%%%%%%%%%%%%% CHAPTER %%%%%%%%%%%%%%%%%%%%%%%%
\chapter{O Oligopólio}

%%%%%%%%%%%%%%%%%%%%%%%% CHAPTER %%%%%%%%%%%%%%%%%%%%%%%%
\chapter{A Teoria dos Jogos}

%%%%%%%%%%%%%%%%%%%%%%%% CHAPTER %%%%%%%%%%%%%%%%%%%%%%%%
\chapter{Aplicações da Teoria dos Jogos}

%%%%%%%%%%%%%%%%%%%%%%%% PART %%%%%%%%%%%%%%%%%%%%%%%%
\part{Tópicos Avançados}

%%%%%%%%%%%%%%%%%%%%%%%% CHAPTER %%%%%%%%%%%%%%%%%%%%%%%%
\chapter{Economia Comportamental}

%%%%%%%%%%%%%%%%%%%%%%%% CHAPTER %%%%%%%%%%%%%%%%%%%%%%%%
\chapter{Trocas}

%%%%%%%%%%%%%%%%%%%%%%%% CHAPTER %%%%%%%%%%%%%%%%%%%%%%%%
\chapter{Produção}

%%%%%%%%%%%%%%%%%%%%%%%% CHAPTER %%%%%%%%%%%%%%%%%%%%%%%%
\chapter{O Bem-Estar}

%%%%%%%%%%%%%%%%%%%%%%%% CHAPTER %%%%%%%%%%%%%%%%%%%%%%%%
\chapter{Externalidades}

%%%%%%%%%%%%%%%%%%%%%%%% CHAPTER %%%%%%%%%%%%%%%%%%%%%%%%
\chapter{Tecnologia da Informação}

%%%%%%%%%%%%%%%%%%%%%%%% CHAPTER %%%%%%%%%%%%%%%%%%%%%%%%
\chapter{Bens Públicos}

%%%%%%%%%%%%%%%%%%%%%%%% CHAPTER %%%%%%%%%%%%%%%%%%%%%%%%
\chapter{Informação Assimétrica}

\end{document}
