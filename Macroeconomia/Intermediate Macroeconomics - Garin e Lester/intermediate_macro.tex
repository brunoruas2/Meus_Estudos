%%%%%%%%%%%%%%%%%%%%%%%%%%%%%%%%%%%%%%%%%%%%%%%%%%%
%% LaTeX book template                           %%
%% Author:  Amber Jain (http://amberj.devio.us/) %%
%% License: ISC license                          %%
%%%%%%%%%%%%%%%%%%%%%%%%%%%%%%%%%%%%%%%%%%%%%%%%%%%

\documentclass[a4paper,11pt]{book}

\usepackage[T1]{fontenc}
\usepackage[utf8]{inputenc}
\usepackage{lmodern}
\usepackage{hyperref}
\usepackage{graphicx}
\usepackage[portuguese]{babel}
\usepackage{amsfonts}
\usepackage[usenames,dvipsnames]{xcolor}
\usepackage{mathtools}
\usepackage{amssymb} % alguns simbolos matematicos
\usepackage{ mathrsfs } % letras cursivas

% para usar definições
\usepackage{amsthm}

\theoremstyle{definition}
\newtheorem{definition}{Definition}[section]

% tirando identação dos paragrafos
\setlength{\parindent}{0ex}

% coding examples in latex file
\usepackage{listings}
\usepackage{xcolor}

\definecolor{codegreen}{rgb}{0,0.6,0}
\definecolor{codegray}{rgb}{0.5,0.5,0.5}
\definecolor{codepurple}{rgb}{0.58,0,0.82}
\definecolor{backcolour}{rgb}{0.95,0.95,0.92}

\lstdefinestyle{mystyle}{
    backgroundcolor=\color{backcolour},   
    commentstyle=\color{codegreen},
    keywordstyle=\color{magenta},
    numberstyle=\tiny\color{codegray},
    stringstyle=\color{codepurple},
    basicstyle=\ttfamily\footnotesize,
    breakatwhitespace=false,         
    breaklines=true,                 
    captionpos=b,                    
    keepspaces=true,                 
    numbers=left,                    
    numbersep=5pt,                  
    showspaces=false,                
    showstringspaces=false,
    showtabs=false,                  
    tabsize=2
}

\lstset{style=mystyle}

%%%%%%%%%%%%%%%%%%%%%%%%%%%%%%%%%%%%%%%%%%%%%%%%%%%%%%%%%%%%%%%%%%%%%%%%%%%%%%%%
% 'dedication' environment: To add a dedication paragraph at the start of book %
% Source: http://www.tug.org/pipermail/texhax/2010-June/015184.html            %
%%%%%%%%%%%%%%%%%%%%%%%%%%%%%%%%%%%%%%%%%%%%%%%%%%%%%%%%%%%%%%%%%%%%%%%%%%%%%%%%
\newenvironment{dedication}
{
   \cleardoublepage
   \thispagestyle{empty}
   \vspace*{\stretch{1}}
   \hfill\begin{minipage}[t]{0.66\textwidth}
   \raggedright
}
{
   \end{minipage}
   \vspace*{\stretch{3}}
   \clearpage
}

%%%%%%%%%%%%%%%%%%%%%%%%%%%%%%%%%%%%%%%%%%%%%%%%
% Chapter quote at the start of chapter        %
% Source: http://tex.stackexchange.com/a/53380 %
%%%%%%%%%%%%%%%%%%%%%%%%%%%%%%%%%%%%%%%%%%%%%%%%
\makeatletter
\renewcommand{\@chapapp}{}% Not necessary...
\newenvironment{chapquote}[2][2em]
  {\setlength{\@tempdima}{#1}%
   \def\chapquote@author{#2}%
   \parshape 1 \@tempdima \dimexpr\textwidth-2\@tempdima\relax%
   \itshape}
  {\par\normalfont\hfill--\ \chapquote@author\hspace*{\@tempdima}\par\bigskip}
\makeatother


%%%%%%%%%%%%%%%%%%%%%%%%%%%%%%%%%%%%%%%%%%%%%%%%%%%
% First page of book which contains 'stuff' like: %
%  - Book title, subtitle                         %
%  - Book author name                             %
%%%%%%%%%%%%%%%%%%%%%%%%%%%%%%%%%%%%%%%%%%%%%%%%%%%
% Book's title and subtitle
\title{\Huge \textbf{Intermediate Macroeconomics} \\ 
\huge }
% Author
\author{\textsc{Garín e Lestar}}


\begin{document}

\frontmatter
\maketitle

\tableofcontents
%\listoffigures
%\listoftables

\mainmatter

\part{Introdução}

%%%%%%%%%%%%%%%%%%%%%%%%%%%%%%%%%%%%%%%%%%%%%%%%%%%
%                 CHAPTER                         %
%%%%%%%%%%%%%%%%%%%%%%%%%%%%%%%%%%%%%%%%%%%%%%%%%%%
\chapter{Dados macroeconômicos}
\chapter{O que é um modelo?}
\chapter{Uma breve história do pensamento macroeconômico}

\part{O Longo Prazo}

\chapter{Fatos sobre crescimento econômico}
\chapter{O modelo básico de Solow}
\chapter{O modelo expandido de Solow}
\chapter{Compreendendo as diferenças de renda entre países}
\chapter{Gerações sobrepostas}

\part{A Microeconomia da Macroeconomia}

\chapter{Um modelo dinâmico de consumo-poupança}
\chapter{Um modelo multiperíodo de consumo-poupança}
\chapter{Equilíbrio em uma endowment economy}
\chapter{Produção, demanda por trabalho, investimento e oferta de trabalho}
\chapter{Política fiscal}
\chapter{Dinheiro}
\chapter{Eficiência do equilíbrio}
\chapter{Competição monopolística}
\chapter{Busca, matching e desemprego}

\part{O Médio Prazo}

\chapter{O modelo neoclássico}
\chapter{Efeitos dos choques no modelo neoclássico}
\chapter{Trazendo o modelo neoclássico aos dados}
\chapter{Dinheiro, inflação e taxa de juros}
\chapter{Implicações políticas e críticas ao modelo neoclássico}
\chapter{Modelo neoclássico para uma economia aberta}

\part{O Curto Prazo}

\chapter{O modelo novo-keynesiano - demanda}
\chapter{O modelo novo-keynesiano - oferta}
\chapter{Efeitos dos choques no modelo novo-keynesiano}
\chapter{Dinâmica no modelo novo-keynesiano: do curto ao longo prazo}
\chapter{Política monetária no modelo novo-keynesiano}
\chapter{O limite inferior zero}
\chapter{Modelo novo-keynesiano para uma economia aberta}

\part{Dinheiro, Crédito, Bancos e Finanças}

\chapter{O básico sobre bancos}
\chapter{O processo de criação de dinheiro}
\chapter{Um modelo de transformação de liquidez e corridas bancárias}
\chapter{Risco e preço dos títulos e Estruturas a termo das taxas de juros}
\chapter{Choques no mercado de ações e bolhas}
\chapter{Fatores financeiros em um modelo macro}
\chapter{Crises financeiras e a Grande Recessão}

\part{Apêndice}

\chapter{Apêndice - Matemática}
\chapter{Apêndice - Probabilidade e Estatística}
\chapter{Apêndice - O modelo neoclássico com uma curva de oferta ascendente}
\chapter{Apêndice - O modelo novo-keynesiano com salários fixos}
\chapter{Apêndice - Substituindo a curva LM pela curva MP}














\end{document}