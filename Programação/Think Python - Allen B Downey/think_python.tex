%%%%%%%%%%%%%%%%%%%%%%%%%%%%%%%%%%%%%%%%%%%%%%%%%%%
%% LaTeX book template                           %%
%% Author:  Amber Jain (http://amberj.devio.us/) %%
%% License: ISC license                          %%
%%%%%%%%%%%%%%%%%%%%%%%%%%%%%%%%%%%%%%%%%%%%%%%%%%%

\documentclass[a4paper,11pt]{book}
\usepackage[T1]{fontenc}
\usepackage[utf8]{inputenc}
\usepackage{lmodern}
\usepackage{hyperref}
\usepackage{graphicx}
\usepackage[portuguese]{babel}

% coding examples in latex file
\usepackage{listings}
\usepackage{xcolor}

\definecolor{codegreen}{rgb}{0,0.6,0}
\definecolor{codegray}{rgb}{0.5,0.5,0.5}
\definecolor{codepurple}{rgb}{0.58,0,0.82}
\definecolor{backcolour}{rgb}{0.95,0.95,0.92}

\lstdefinestyle{mystyle}{
    backgroundcolor=\color{backcolour},   
    commentstyle=\color{codegreen},
    keywordstyle=\color{magenta},
    numberstyle=\tiny\color{codegray},
    stringstyle=\color{codepurple},
    basicstyle=\ttfamily\footnotesize,
    breakatwhitespace=false,         
    breaklines=true,                 
    captionpos=b,                    
    keepspaces=true,                 
    numbers=left,                    
    numbersep=5pt,                  
    showspaces=false,                
    showstringspaces=false,
    showtabs=false,                  
    tabsize=2
}

\lstset{style=mystyle}


%%%%%%%%%%%%%%%%%%%%%%%%%%%%%%%%%%%%%%%%%%%%%%%%
% Chapter quote at the start of chapter        %
% Source: http://tex.stackexchange.com/a/53380 %
%%%%%%%%%%%%%%%%%%%%%%%%%%%%%%%%%%%%%%%%%%%%%%%%
\makeatletter
\renewcommand{\@chapapp}{}% Not necessary...
\newenvironment{chapquote}[2][2em]
  {\setlength{\@tempdima}{#1}%
   \def\chapquote@author{#2}%
   \parshape 1 \@tempdima \dimexpr\textwidth-2\@tempdima\relax%
   \itshape}
  {\par\normalfont\hfill--\ \chapquote@author\hspace*{\@tempdima}\par\bigskip}
\makeatother


% tirando identação dos paragrafos
\setlength{\parindent}{0ex}

%%%%%%%%%%%%%%%%%%%%%%%%%%%%%%%%%%%%%%%%%%%%%%%%%%%
% First page of book which contains 'stuff' like: %
%  - Book title, subtitle                         %
%  - Book author name                             %
%%%%%%%%%%%%%%%%%%%%%%%%%%%%%%%%%%%%%%%%%%%%%%%%%%%

% Book's title and subtitle
\title{\Huge \textbf{Think Python}  \\ \huge How to think like a computer scientist }
% Author
\author{\textsc{Allen B. Downey}}


\begin{document}

\frontmatter
\maketitle

\tableofcontents
%\listoffigures
%\listoftables

\mainmatter

%%%%%%%%%%%%%%%%%%%%%%%%%%%%%%%%%%%%%%%%%%%%%%%%%%%
%                 CHAPTER                         %
%%%%%%%%%%%%%%%%%%%%%%%%%%%%%%%%%%%%%%%%%%%%%%%%%%%
\chapter{The Way of the Program}

\begin{chapquote}{página 1}
	``The single most important skill for a computer scientistis \textbf{problem solving}. That is the 
	ability to formulate problems, think creatively about solutions, and express a solution clearly and 
	accurately''
\end{chapquote}

\section{O que é um Programa?}
Um \textbf{programa} é uma sequencia de instruções que especifica como fazer um determinado computation. Um \textbf{computation}
\footnote{\href{https://en.wikipedia.org/wiki/Computation}{https://en.wikipedia.org/wiki/Computation}} é qualquer tipo de cálculo que inclua 
passos artiméticos (computação matemática) e não artiméticos (computação simbólica) e que segue uma determinada ordem. \par
Algumas características comuns em qualquer linguagem de programação:
\begin{itemize}
	\item input $\rightarrow$ dados de teclado, arquivo, rede, etc.
	\item output $\rightarrow$ mostrar dados na tela, salvar em um arquivo, enviar na rede, etc.
	\item math $\rightarrow$ operações matemáticas
	\item conditional execution $\rightarrow$ só executar um pedaço do código após certas condições
	\item repetition $\rightarrow$ reexecutar um pedaço do código com alguma variação
\end{itemize}

\section{Executando o Python}
O \textbf{interpretador} do Python é um programa que lê e executa códigos escritos em Python. Tem duas maneiras de se executar códigos em Python
a primeira é no modo \textbf{interativo} onde podemos enviar as ordens direto pro interpretador e a segunda em em modo \textbf{script} onde
o interpretador analisa o código inteiro antes de começar o processamento.

\section{Operadores Aritméticos}
\begin{lstlisting}[language=Python, caption=Operadores Básicos]
# Soma
40 + 1

# Subtracao
10 - 1

# Multiplicacao
20 * 10

# Divisao
11 / 1

# Exponenciacao
6 ** 2

\end{lstlisting}

\section{Linguagens Formais e Naturais}

As \textbf{linguagens naturais} são as linguagens que as pessoas usam para se comunicar no dia a dia. São criadas e se modificam organicamente com o passar do tempo.

Por outro lado, as \textbf{linguagens formais} são criadas por pessoas e possuem uma aplicação específica. As linguagens de computação são linguagens formais desenvolvidas para expressar computações.

As linguagens formais possuem regras estritas que chamamos de \textbf{regras de sintaxe}.

\subsection{Regras de Sintaxe}

As regras de sintaxe possuem dois sentidos gerais:
\begin{itemize}
	\item tokens $\rightarrow$ são os elementos da linguagem
	\item estrutura $\rightarrow$ é a regra de combinação dos tokens
\end{itemize}

Quando o interpretador lê um código, ele faz um processo chamado \textbf{parsing} que é justamente entender a estrutura de combinação entre os diferentes tokens presentes no script. No caso do Python a identação faz um papel importante na estrutura.

\section{Debugging}

Os erros em um programa são chamados de \textbf{bugs} e o processo de resolução desses erros é chamado de \textbf{debugging}.

\section{Glossário}
\begin{itemize}
	\item problem solving $\rightarrow$ processo de formular um problema, achar uma solução e expressá-la
	\item high-level language $\rightarrow$ lingaugem desenhada para ser fácil para um humano ler e escrever
	\item low-level language $\rightarrow$ lingaugem desenhada para ser fácil para um computador ler. AKA "machine language" ou "assembly language"
	\item portability $\rightarrow$ capacidade de um programa rodar em mais de um tipo de computador
	\item interpreter $\rightarrow$ um programa que lê outro programa e o executa
	\item prompt $\rightarrow$ caracteres no prompt que indicam que ele está pronto para receber input direto do usuário
	\item program $\rightarrow$ um conjunto de instruções que especifica uma computation
	\item print statement $\rightarrow$ uma instrução que diz para o interpretador mostrar caracteres na tela
	\item operator $\rightarrow$ um caracter especial que representa uma computação simples (+, -, *, /, ** ) 
	\item value $\rightarrow$ uma das unidades básica de dados
	\item type $\rightarrow$ a categoria em que um valor pertence
	\item interger $\rightarrow$ numero inteiros
	\item floating-point $\rightarrow$ números fracionados
	\item string $\rightarrow$ sequencia de caracteres
	\item natural language $\rightarrow$ linguagem humana
	\item formal language $\rightarrow$ linguagem com objetivo e regras específicos
	\item token $\rightarrow$ um dos elementos básicos de uma linguagem formal, análogo à palavra na linguagem natural
	\item syntax $\rightarrow$ as regras de estrutura de um programa
	\item parse $\rightarrow$ processo de examinar um programa e analisar sua estrutura
	\item bug $\rightarrow$ um erro no programa
	\item debugg $\rightarrow$ o processo de resolver um erro
\end{itemize}

%%%%%%%%%%%%%%%%%%%%%%%%%%%%%%%%%%%%%%%%%%%%%%%%%%%
%                 CHAPTER                         %
%%%%%%%%%%%%%%%%%%%%%%%%%%%%%%%%%%%%%%%%%%%%%%%%%%%
\chapter{Varibles, Expressions and Statements}

\begin{chapquote}{página 11}
	``One of the most powerful features of a programming language is the ability to manipulate \textbf{variables}. A variable is a name that refers to a value''
\end{chapquote}

\section{Statements de atribuição}
\begin{lstlisting}[language=Python, caption=Atribuições para diferentes classes]
message = 'acesse www.economiamainstream.com.br'

n = 12

pi = 3.1415
\end{lstlisting}











%%%%%%%%%%%%%%%%%%%%%%%%%%%%%%%%%%%%%%%%%%%%%%%%%%%
%                 CHAPTER                         %
%%%%%%%%%%%%%%%%%%%%%%%%%%%%%%%%%%%%%%%%%%%%%%%%%%%
\chapter{Functions}
\chapter{Case Study: Interface Design}
\chapter{conditionals and Recursion}
\chapter{Fruitful Funtions}
\chapter{Iteration}
\chapter{Strings}
\chapter{Case Study: Word Play}
\chapter{Lists}
\chapter{Dictionaries}
\chapter{Tuples}
\chapter{Case Study: Data Structure Selection}
\chapter{Files}
\chapter{Classes and Objects}
\chapter{Classes and Functions}
\chapter{Classes and Methods}
\chapter{Inheritance}
\chapter{The Goodies}
\chapter{Debugging}
\chapter{Analysis of Algorithms}


\end{document}