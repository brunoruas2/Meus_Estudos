%%%%%%%%%%%%%%%%%%%%%%%%%%%%%%%%%%%%%%%%%%%%%%%%%%%
%% LaTeX book template                           %%
%% Author:  Amber Jain (http://amberj.devio.us/) %%
%% License: ISC license                          %%
%%%%%%%%%%%%%%%%%%%%%%%%%%%%%%%%%%%%%%%%%%%%%%%%%%%

\documentclass[a4paper,11pt]{book}
\usepackage[T1]{fontenc}
\usepackage[utf8]{inputenc}
\usepackage{lmodern}

%%%%%%%%%%%%%%%%%%%%%%%%%%%%%%%%%%%%%%%%%%%%%%%%%%%%%%%%%
% Source: http://en.wikibooks.org/wiki/LaTeX/Hyperlinks %
%%%%%%%%%%%%%%%%%%%%%%%%%%%%%%%%%%%%%%%%%%%%%%%%%%%%%%%%%

\usepackage{hyperref}
\usepackage{graphicx}
\usepackage[portuguese]{babel}

%%%%%%%%%%%%%%%%%%%%%%%%%%%%%%%%%%%%%%%%%%%%%%%%%%%%%%%%%%%%%%%%%%%%%%%%%%%%%%%%
% 'dedication' environment: To add a dedication paragraph at the start of book %
% Source: http://www.tug.org/pipermail/texhax/2010-June/015184.html            %
%%%%%%%%%%%%%%%%%%%%%%%%%%%%%%%%%%%%%%%%%%%%%%%%%%%%%%%%%%%%%%%%%%%%%%%%%%%%%%%%

\newenvironment{dedication}
{
   \cleardoublepage
   \thispagestyle{empty}
   \vspace*{\stretch{1}}
   \hfill\begin{minipage}[t]{0.66\textwidth}
   \raggedright
}
{
   \end{minipage}
   \vspace*{\stretch{3}}
   \clearpage
}

%%%%%%%%%%%%%%%%%%%%%%%%%%%%%%%%%%%%%%%%%%%%%%%%
% Chapter quote at the start of chapter        %
% Source: http://tex.stackexchange.com/a/53380 %
%%%%%%%%%%%%%%%%%%%%%%%%%%%%%%%%%%%%%%%%%%%%%%%%

\makeatletter
\renewcommand{\@chapapp}{}% Not necessary...
\newenvironment{chapquote}[2][2em]
  {\setlength{\@tempdima}{#1}%
   \def\chapquote@author{#2}%
   \parshape 1 \@tempdima \dimexpr\textwidth-2\@tempdima\relax%
   \itshape}
  {\par\normalfont\hfill--\ \chapquote@author\hspace*{\@tempdima}\par\bigskip}
\makeatother

%%%%%%%%%%%%%%%%%%%%%%%%%%%%%%%%%%%%%%%%%%%%%%%%%%%
% First page of book which contains 'stuff' like: %
%  - Book title, subtitle                         %
%  - Book author name                             %
%%%%%%%%%%%%%%%%%%%%%%%%%%%%%%%%%%%%%%%%%%%%%%%%%%%

% Book's title and subtitle
\title{\Huge \textbf{\LaTeX \ in 24 Hours} \\ 
\huge Datta Dilip}

% Author
\author{\textsc{Bruno de M. Ruas}}


\begin{document}

\frontmatter
\maketitle

%%%%%%%%%%%%%%%%%%%%%%%%%%%%%%%%%%%%%%%%%%%%%%%%%%%%%%%%%%%%%%%%%%%%%%%%
% Auto-generated table of contents, list of figures and list of tables %
%%%%%%%%%%%%%%%%%%%%%%%%%%%%%%%%%%%%%%%%%%%%%%%%%%%%%%%%%%%%%%%%%%%%%%%%
\tableofcontents

\mainmatter

\chapter{Introdução}
\section{Partes de um \LaTeX \ file}
A estrutura principal de um arquivo de input pode ser divida em duas partes: \textbf{preamble} e \textbf{body}

\subsection{Preamble}
É a parte que contém as configurações globais de processamento como tipo de documento (dtype),
formatação da página, cabeçalho e rodapé, inclusão de packages e definições de instruções
adicionais.

\subsubsection{Tipos de Documentos (dtpye)}
As classes (ou tipos) de documentos default são: \textbf{letter}, \textbf{article}, \textbf{report} ou \textbf{book}.

\subsection{Sintaxe}
\subsubsection{Comandos}
Os comando em \LaTeX \ possuem 3 elementos $\backslash[\ ]\{\}$, i.e., uma \textbf{barra}, \textbf{instruções adicionais} (em colchetes) e seus \textbf{argumentos} (entre chaves).

\subsubsection{Ambientes}
Um ambiente começa com o comando $\backslash begin\{ nome \  do \ ambiente \}$ e terminam com 
$\backslash end\{ nome \  do \ ambiente \}$

\subsubsection{Pacotes}
Um pacote é inserido entre o preamble ($\backslash$documentclass\{\}) e o body \\ ($\backslash$begin\{document\}).
Para carregar um pacote, basta usar o comando \\ $\backslash$usepackage\{nome\_pacote\}

\chapter{Fontes}
\chapter{Formatação de Textos I}
\chapter{Formatação de Textos II}
\chapter{Layout e Estilo da Página}
\chapter{Listas e Tabulação}
\chapter{Tabelas I}
\chapter{Tabelas II}
\chapter{Inserindo Figuras}
\chapter{Desenhos}
\chapter{Equações I}
\chapter{Equações II}
\chapter{Macros}
\chapter{Bibliografia com \LaTeX}
\chapter{Bibliografia com BibTex}
\chapter{Sumário e Index}
\chapter{Miscellaneous I}
\chapter{Miscellaneous II}
\chapter{Carta e Artigo}
\chapter{Livro e Relatório}
\chapter{Slide I}
\chapter{Slide II}
\chapter{Errors e Warnings}
\chapter{Exercícios}


\end{document}